
4.2.1 Conceptual slippery slope
It may be true that there is no essential difference between 499 grains of sand
and 500 grains of sand. But even if that is so, it doesn’t follow that there is no
difference between 1 grain of sand and 5 billion grains of sand. In general, just
because we cannot draw a distinction between A and B, and we cannot draw a
distinction between B and C, it doesn’t mean we cannot draw a distinction
between A and C. Here is an example of a conceptual slippery slope fallacy.
It is illegal for anyone under 21 to drink alcohol. But there is no
difference between someone who is 21 and someone who is 20 years 11
months old. So there is nothing wrong with someone who is 20 years and
11 months old drinking. But since there is no real distinction between
being one month older and one month younger, there shouldn’t be
anything wrong with drinking at any age. Therefore, there is nothing
wrong with allowing a 10 year old to drink alcohol.

Imagine the life of an individual in stages of 1 month intervals. Even if it is true
that there is no distinction in kind between any one of those stages, it doesn’t
follow that there isn’t a distinction to be drawn at the extremes of either end.
Clearly there is a difference between a 5 year old and a 25 year old—a
distinction in kind that is relevant to whether they should be allowed to drink
alcohol. The conceptual slippery slope fallacy assumes that because we cannot
draw a distinction between adjacent stages, we cannot draw a distinction at all
between any stages. One clear way of illustrating this is with color. Think of a
color spectrum from purple to red to orange to yellow to green to blue. Each
color grades into the next without there being any distinguishable boundaries
between the colors—a continuous spectrum. Even if it is true that for any two
adjacent hues on the color wheel, we cannot distinguish between the two, it
doesn’t follow from this that there is no distinction to be drawn between any two
portions of the color wheel, because then we’d be committed to saying that
there is no distinguishable difference between purple and yellow! The example
of the color spectrum illustrates the general point that just because the
boundaries between very similar things on a spectrum are vague, it doesn’t
follow that there are no differences between any two things on that spectrum.
Whether or not one will identify an argument as committing a conceptual
slippery slope fallacy, depends on the other things one believes about the world.
Thus, whether or not a conceptual slippery slope fallacy has been committed will
often be a matter of some debate. It will itself be vague. Here is a good
example that illustrates this point.

People are found not guilty by reason of insanity when they cannot avoid
breaking the law. But people who are brought up in certain deprived
social circumstances are not much more able than the legally insane to
avoid breaking the law. So we should not find such individuals guilty any
more than those who are legally insane.

Whether there is conceptual slippery slope fallacy here depends on what you
think about a host of other things, including individual responsibility, free will,
the psychological and social effects of deprived social circumstances such as
poverty, lack of opportunity, abuse, etc. Some people may think that there are
big differences between those who are legally insane and those who grow up in
deprived social circumstances. Others may not think the differences are so
great. The issues here are subtle, sensitive, and complex, which is why it is
difficult to determine whether there is any fallacy here or not. If the differences
between those who are insane and those who are the product of deprived social
circumstances turn out to be like the differences between one shade of yellow
and an adjacent shade of yellow, then there is no fallacy here. But if the
differences turn out to be analogous to those between yellow and green (i.e.,
with many distinguishable stages of difference between) then there would
indeed be a conceptual slippery slope fallacy here.

The difficulty of
distinguishing instances of the conceptual slippery slope fallacy, and the fact
that distinguishing it requires us to draw on our knowledge about the world,
shows that the conceptual slippery slope fallacy is an informal fallacy.

196

Chapter 4: Informal fallacies

4.2.2 Causal slippery slope fallacy
The causal slippery slope fallacy is committed when one event is said to lead to
some other (usually disastrous) event via a chain of intermediary events. If you
have ever seen Direct TV’s “get rid of cable” commercials, you will know exactly
what I’m talking about. (If you don’t know what I’m talking about you should
Google it right now and find out. They’re quite funny.) Here is an example of a
causal slippery slope fallacy (it is adapted from one of the Direct TV
commercials):

If you use cable, your cable will probably go on the fritz. If your cable is
on the fritz, you will probably get frustrated. When you get frustrated you
will probably hit the table. When you hit the table, your young daughter
will probably imitate you. When your daughter imitates you, she will
probably get thrown out of school. When she gets thrown out of school,
she will probably meet undesirables. When she meets undesirables, she
will probably marry undesirables. When she marries undesirables, you
will probably have a grandson with a dog collar. Therefore, if you use
cable, you will probably have a grandson with dog collar.

This example is silly and absurd, yes. But it illustrates the causal slippery slope
fallacy. Slippery slope fallacies are always made up of a series of conjunctions of
probabilistic conditional statements that link the first event to the last event. A
causal slippery slope fallacy is committed when one assumes that just because
each individual conditional statement is probable, the conditional that links the
first event to the last event is also probable. Even if we grant that each “link” in
the chain is individually probable, it doesn’t follow that the whole chain (or the
conditional that links the first event to the last event) is probable. Suppose, for
the sake of the argument, we assign probabilities to each “link” or conditional
statement, like this. (I have italicized the consequents of the conditionals and
assigned high conditional probabilities to them. The high probability is for the
sake of the argument; I don’t actually think these things are as probable as I’ve
assumed here.)

If you use cable, then your cable will probably go on the fritz (.9)
If your cable is on the fritz, then you will probably get angry (.9)
If you get angry, then you will probably hit the table (.9)
If you hit the table, your daughter will probably imitate you (.8)

197

Chapter 4: Informal fallacies

If your daughter imitates you, she will probably be kicked out of school
(.8)
If she is kicked out of school, she will probably meet undesirables (.9)
If she meets undesirables, she will probably marry undesirables (.8)
If she marries undesirables, you will probably have a grandson with a dog
collar (.8)
However, even if we grant the probabilities of each link in the chain is high (8090% probable), the conclusion doesn’t even reach a probability higher than
chance. Recall that in order to figure the probability of a conjunction, we must
multiply the probability of each conjunct:
(.9) × (.9) × (.9) × (.8) × (.8) × (.9) × (.8) × (.8) = .27
That means the probability of the conclusion (i.e., that if you use cable, you will
have a grandson with a dog collar) is only 27%, despite the fact that each
conditional has a relatively high probability! The causal slippery slope fallacy is
actually a formal probabilistic fallacy and so could have been discussed in
chapter 3 with the other formal probabilistic fallacies. What makes it a formal
rather than informal fallacy is that we can identify it without even having to know
what the sentences of the argument mean. I could just have easily written out a
nonsense argument comprised of series of probabilistic conditional statements.
But I would still have been able to identify the causal slippery slope fallacy
because I would have seen that there was a series of probabilistic conditional
statements leading to a claim that the conclusion of the series was also probable.
That is enough to tell me that there is a causal slippery slope fallacy, even if I
don’t really understand the meanings of the conditional statements.
It is helpful to contrast the causal slippery slope fallacy with the valid form of
inference, hypothetical syllogism. Recall that a hypothetical syllogism has the
following kind of form:
A⊃B
B⊃C
C⊃D
D⊃E
∴ A⊃E

198

Chapter 4: Informal fallacies

The only difference between this and the causal slippery slope fallacy is that
whereas in the hypothetical syllogism, the link between each component is
certain, in a causal slippery slope fallacy, the link between each event is
probabilistic. It is the fact that each link is probabilistic that accounts for the
fallacy. One way of putting this is point is that probability is not transitive. Just
because A makes B probable and B makes C probable and C makes X probable,
it doesn’t follow that A makes X probable. In contrast, when the links are certain
rather than probable, then if A always leads to B and B always leads to C and C
always leads to X, then it has to be the case that A always leads to X.

4.3 Fallacies of relevance
What all fallacies of relevance have in common is that they make an argument or
response to an argument that is irrelevant. Fallacies of relevance can be
compelling psychologically, but it is important to distinguish between rhetorical
techniques that are psychologically compelling, on the one hand, and rationally
compelling arguments, on the other. What makes something a fallacy is that it
fails to be rationally compelling, once we have carefully considered it. That said,
arguments that fail to be rationally compelling may still be psychologically or
emotionally compelling. The first fallacy of relevance that we will consider, the
ad hominem fallacy, is an excellent example a fallacy that can be psychologically
compelling.
4.3.1 Ad hominem
“Ad hominem” is a Latin phrase that can be translated into English as the phrase,
“against the man.” In an ad hominem fallacy, instead of responding to (or
attacking) the argument a person has made, one attacks the person him or
herself. In short, one attacks the person making the argument rather than the
argument itself. Here is an anecdote that reveals an ad hominem fallacy (and
that has actually occurred in my ethics class before).
A philosopher named Peter Singer had made an argument that it is
morally wrong to spend money on luxuries for oneself rather than give all
of your money that you don’t strictly need away to charity. The argument
is actually an argument from analogy (whose details I discussed in section
3.3), but the essence of the argument is that there are every day in this
world children who die preventable deaths, and there are charities who

199


