Chapter 3: Evaluating inductive arguments and probabilistic and statistical fallacies

3.3 Analogical arguments
Another kind of common inductive argument is an argument from analogy. In
an argument from analogy, we note that since some thing x shares similar
properties to some thing y, then since y has characteristic A, x probably has
characteristic A as well. For example, suppose that I have always owned Subaru
cars in the past and that they have always been reliable and I argue that the new
car I’ve just purchased will also be reliable because it is a Subaru. The two
things in the analogy are 1) the Subarus I have owned in the past and 2) the
current Subaru I have just purchased. The similarity between these two things is
just that they are both Subarus. Finally, the conclusion of the argument is that
this Subaru will share the characteristic of being reliable with the past Subarus I
have owned. Is this argument a strong or weak inductive argument? Partly it
depends on how many Subarus I’ve owned in the past. If I’ve only owned one,
then the inference seems fairly weak (perhaps I was just lucky in that one Subaru
I’ve owned). If I’ve owned ten Subarus then the inference seems much stronger.
Thus, the reference class that I’m drawing on (in this case, the number of
Subarus I’ve previously owned) must be large enough to generalize from
(otherwise we would be committing the fallacy of “hasty generalization”).
However, even if our reference class was large enough, what would make the
inference even stronger is knowing not simply that the new car is a Subaru, but
also specific things about its origin. For example, if I know that this particular
model has the same engine and same transmission as the previous model I
owned and that nothing significant has changed in how Subarus are made in the
intervening time, then my argument is strengthened. In contrast, if this new
Subaru was made after Subaru was bought by some other car company, and if
the engine and transmission were actually made by this new car company, then
my argument is weakened. It should be obvious why: the fact that the car is still
called “Subaru” is not relevant establishing that it will have the same
characteristics as the other cars that I’ve owned that were called “Subarus.”
Clearly, what the car is called has no inherent relevance to whether the car is
reliable. Rather, what is relevant to whether the car is reliable is the quality of
the parts and assembly of the car. Since it is possible that car companies can
retain their name and yet drastically alter the quality of the parts and assembly
of the car, it is clear that the name of the car isn’t itself what establishes the
quality of the car. Thus, the original argument, which invoked merely that the
new car was a Subaru is not as strong as the argument that the car was

154

Chapter 3: Evaluating inductive arguments and probabilistic and statistical fallacies

constructed with the same quality parts and quality assembly as the other cars
I’d owned (and that had been reliable for me). What this illustrates is that better
arguments from analogy will invoke more relevant similarities between the
things being compared in the analogy. This is a key condition for any good
argument from analogy: the similar characteristics between the two things cited
in the premises must be relevant to the characteristic cited in the conclusion.
Here is an ethical argument that is an argument from analogy.1 Suppose that
Bob uses his life savings to buy an expensive sports car. One day Bob parks his
car and takes a walk along a set of train tracks. As he walks, he sees in the
distance a small child whose leg has become caught in the train tracks. Much to
his alarm, he sees a train coming towards the child. Unfortunately, the train will
reach the child before he can (since it is moving very fast) and he knows it will be
unable to stop in time and will kill the child. At just that moment, he sees a
switch near him that he can throw to change the direction of the tracks and
divert the train onto another set of tracks so that it won’t hit the child.
Unfortunately, Bob sees that he has unwittingly parked his car on that other set
of tracks and that if he throws the switch, his expensive car will be destroyed.
Realizing this, Bob decides not to throw the switch and the train strikes and kills
the child, leaving his car unharmed. What should we say of Bob? Clearly, that
was a horrible thing for Bob to do and we would rightly judge him harshly for
doing it. In fact, given the situation described, Bob would likely be criminally
liable. Now consider the following situation in which you, my reader, likely find
yourself (whether you know it or not—well, now you do know it). Each week you
spend money on things that you do not need. For example, I sometimes buy $5
espressos from Biggby’s or Starbuck’s. I do not need to have them and I could
get a much cheaper caffeine fix, if I chose to (for example, I could make a strong
cup of coffee at my office and put sweetened hazelnut creamer in it). In any
case, I really don’t need the caffeine at all! And yet I regularly purchase these $5
drinks. (If $5 drinks aren’t the thing you spend money on, but in no way need,
then fill in the example with whatever it is that fits your own life.) With the
money that you could save from forgoing these luxuries, you could, quite
literally, save a child’s life. Suppose (to use myself as an example) I were to buy
two $5 coffees a week (a conservative estimate). That is $10 a week, roughly
$43 a month and $520 a year. Were I to donate that amount (just $40/month) to
an organization such as the Against Malaria Foundation, I could save a child’s
1

This argument comes (with interpretive liberties on my part) from Peter Singer’s, “The Singer
Solution to World Poverty” published in the NY Times Magazine, September 5, 1999.

155

Chapter 3: Evaluating inductive arguments and probabilistic and statistical fallacies

life in just six years.2 Given these facts, and comparing these two scenarios
(Bob’s and your own), the argument from analogy proceeds like this:
1. Bob chose to have a luxury item for himself rather than to save the life
of a child.
2. “We” regularly choose having luxury items rather than saving the life
of a child.
3. What Bob did was morally wrong.
4. Therefore, what we are doing is morally wrong as well.
The two things being compared here are Bob’s situation and our own. The
argument then proceeds by claiming that since we judge what Bob did to be
morally wrong, and since our situation is analogous to Bob’s in relevant respects
(i.e., choosing to have luxury items for ourselves rather than saving the lives of
dying children), then our actions of purchasing luxury items for ourselves must
be morally wrong for the same reason.
One way of arguing against the conclusion of this argument is by trying to argue
that there are relevant disanalogies between Bob’s situation and our own. For
example, one might claim that in Bob’s situation, there was something much
more immediate he could do to save the child’s life right then and there. In
contrast, our own situation is not one in which a child that is physically proximate
to us is in imminent danger of death, where there is something we can
immediately do about it. One might argue that this disanalogy is enough to
show that the two situations are not analogous and that, therefore, the
conclusion does not follow. Whether or not this response to the argument is
adequate, we can see that the way of objecting to an argument from analogy is
by trying to show that there are relevant differences between the two things
being compared in the analogy. For example, to return to my car example,
even if the new car was a Subaru and was made under the same conditions as all
of my other Subarus, if I purchased the current Subaru used, whereas all the
other Subarus had been purchased new, then that could be a relevant difference
that would weaken the conclusion that this Subaru will be reliable.
So we’ve seen that an argument from analogy is strong only if the following two
conditions are met:

2

http://www.givewell.org/giving101/Your-dollar-goes-further-overseas

156

Chapter 3: Evaluating inductive arguments and probabilistic and statistical fallacies

1. The characteristics of the two things being compared must be similar
in relevant respects to the characteristic cited in the conclusion.
2. There must not be any relevant disanalogies between the two things
being compared.
Arguments from analogy that meet these two conditions will tend to be stronger
inductive arguments.
Exercise 24: Evaluate the following arguments from analogy as either
strong or weak. If the argument is weak, cite what you think would be a
relevant disanalogy.
1. Every painting by Rembrandt contains dark colors and illuminated
faces, therefore the original painting that hangs in my high school is
probably by Rembrandt, since it contains dark colors and illuminated
faces.
2. I was once bitten by a poodle. Therefore, this poodle will probably
bite me too.
3. Every poodle I’ve ever met has bitten me (and I’ve met over 300
poodles). Therefore this poodle will probably bite me too.
4. My friend took Dr. Van Cleave’s logic class last semester and got an A.
Since Dr. Van Cleave’s class is essentially the same this semester and
since my friend is no better a student than I am, I will probably get an
A as well.
5. Bill Cosby used his power and position to seduce and rape women.
Therefore, Bill Cosby probably also used his power to rob banks.
6. Every car I’ve ever owned had seats, wheels and brakes and was also
safe to drive. This used car that I am contemplating buying has seats,
wheels and brakes. Therefore, this used car is probably safe to drive.
7. Every Volvo I’ve ever owned was a safe car to drive. My new car is a
Volvo. Therefore, my new car is probably safe to drive.
8. Dr. Van Cleave did not give Jones an excused absence when Jones
missed class for his grandmother’s funeral. Mary will have to miss
class to attend her aunt’s funeral. Therefore, Dr. Van Cleave should
not give Mary an excused absence either.
9. Dr. Van Cleave did not give Jones an excused absence when Jones
missed class for his brother’s birthday party. Mary will have to miss
class to attend her aunt’s funeral. Therefore, Dr. Van Cleave should
not give Mary an excused absence either.

157

Chapter 3: Evaluating inductive arguments and probabilistic and statistical fallacies

10. If health insurance companies pay for heart surgery and brain surgery,
which can both increase an individual’s happiness, then they should
also pay for cosmetic surgery, which can also increase an individual’s
happiness.
11. A knife is an eating utensil that can cut things. A spoon is also an
eating utensil. So a spoon can probably cut things as well.
12. Any artificial, complex object like a watch or a telescope has been
designed by some intelligent human designer. But naturally occurring
objects like eyes and brains are also very complex objects. Therefore,
complex naturally occurring objects must have been designed by
some intelligent non-human designer.
13. The world record holding runner, Kenenisa Bekele ran 100 miles per
week and twice a week did workouts comprised of ten mile repeats on
the track in the weeks leading up to his 10,000 meter world record. I
have run 100 miles per week and have been doing ten mile repeats
twice a week. Therefore, the next race I will run will probably be a
world record.
14. I feel pain when someone hits me in the face with a hockey puck. We
are both human beings, so you also probably feel pain when you are
hit in the face with a hockey puck.
15. The color I experience when I see something as “green” has a
particular quality (that is difficult to describe). You and I are both
human beings, so the color you experience when you see something
green probably has the exact same quality. (That is, what you and I
experience when we see something green is the exact same
experiential color.)

3.4 Causal reasoning
When I strike a match it will produce a flame. It is natural to take the striking of
the match as the cause that produces the effect of a flame. But what if the
matchbook is wet? Or what if I happen to be in a vacuum in which there is no
oxygen (such as in outer space)? If either of those things is the case, then the
striking of the match will not produce a flame. So it isn’t simply the striking of
the match that produces the flame, but a combination of the striking of the
match together with a number of other conditions that must be in place in order
for the striking of the match to create a flame. Which of those conditions we call
the “cause” depends in part on the context. Suppose that I’m in outer space

158

