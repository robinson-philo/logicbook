%module 6-1 pg 153
\subsection{Inductive Logics}
Back in Chapter 1, we made a distinction between deductive and inductive arguments. While
deductive arguments attempt to provide premises that guarantee their conclusions, inductive
arguments are less ambitious. They merely aim to provide premises that make the conclusion more
probable. Because of this difference, it is inappropriate to evaluate deductive and inductive
arguments by the same standards. We do not use the terms `valid' and `invalid' when evaluating
inductive arguments: technically, they're all invalid because their premises don't guarantee their
conclusions; but that's not a fair evaluation, since inductive arguments don't even pretend to try
to provide such a guarantee. Rather, we say of inductive arguments that they are strong or weak--
the more probable the conclusion in light of the premises, the stronger the inductive argument; the
less probable the conclusion, the weaker. These judgments can change in light of new information.
Additional evidence may have the effect of making the conclusion more or less probable--of
strengthening or weakening the argument.

The topic of this chapter and the next will be inductive logic: we will be learning about the various
types of inductive arguments and how to evaluate them. Inductive arguments are a rather motley
bunch. They come in a wide variety of forms that can vary according to subject matter; they resist
the uniform treatment we were able to provide for their deductive cousins. We will have to examine
a wide variety of approaches--different inductive logics. While all inductive arguments have in
common that they attempt to give their conclusions more probable, it is not always possible for us
to make precise judgments about exactly how probable their conclusions are in light of their
premises. When that is the case, we will make relative judgments: this argument is stronger or
weaker than that argument, though I can't say how much stronger or weaker, precisely. Sometimes,
however, it will be possible to render precise judgments about the probability of conclusions, so it
will be necessary for us to acquire basic skills in calculating probabilities. With those in hand, we
will be in a position to model an ideally rational approach to revising our judgments about the
strength of inductive arguments in light of new evidence. In addition, since so many inductive
arguments use statistics, it will be necessary for us to acquire a basic understanding of some
fundamental statistical concepts. With these in hand, we will be in a position to recognize the most
common types of statistical fallacies--mistakes and intentionally misleading arguments that use
statistics to lead us astray.

Probability and statistics will be the subject of a future chapter. In this chapter, we will look at two
very common types of inductive reasoning: arguments from analogy and inferences involving
causation. The former are quite common in everyday life; the latter are the primary methods of
scientific and medical research. Each type of reasoning exhibits certain patterns, and we will look
at the general forms analogical and causal arguments; we want to develop the skill of recognizing
how particular instances of reasoning fit these general patterns. We will also learn how these types
of arguments are evaluated. For arguments from analogy, we will identify the criteria that we use
to make relative judgments about strength and weakness. For causal reasoning, we will compare
the various forms of inference to identify those most likely to produce reliable results, and we will
examine some of the pitfalls peculiar to each that can lead to errors.

\subsection{Arguments from Analogy}
Analogical reasoning is ubiquitous in everyday life. We rely on analogies--similarities between
present circumstances and those we've already experienced--to guide our actions. We use
comparisons to familiar people, places, and things to guide our evaluations of novel ones. We
criticize people's arguments based on their resemblance to obviously absurd lines of reasoning.

In this section, we will look at the various uses of analogical reasoning. Along the way, we will
identify a general pattern that all arguments from analogy follow and learn how to show that
particular arguments fit the pattern. We will then turn to the evaluation of analogical arguments:
we will identify six criteria that govern our judgments about the relative strength of these
arguments. Finally, we will look at the use of analogies to refute other arguments.

\subsubsection{The Form of Analogical Arguments}
Perhaps the most common use of analogical reasoning is to predict how the future will unfold
based on similarities to past experiences. Consider this simple example. When I first learned that
the movie The Wolf of Wall Street was coming out, I predicted that I would like it. My reasoning
went something like this:

\begin{quote}
The Wolf of Wall Street is directed by Martin Scorsese, and it stars Leonardo DiCaprio.
Those two have collaborated several times in the past, on Gangs of New York, The Aviator,
The Departed, and Shutter Island. I liked each of those movies, so I predict that I will like
The Wolf of Wall Street.
\end{quote}

Notice, first, that this is an inductive argument. The conclusion, that I will like The Wolf of Wall
Street is not guaranteed by the premises; as a matter of fact, my prediction was wrong and I really
didn't care for the film. But our real focus here is on the fact that the prediction was made on the
basis of an analogy. Actually, several analogies, between The Wolf of Wall Street, on the one hand,
and all the other Scorsese/DiCaprio collaborations on the other. The new film is similar in
important respects to the older ones; I liked all of those; so, I'll probably like the new one.

We can use this pattern of reasoning for more overtly persuasive purposes. Consider the following:

\begin{quote}
Eating pork is immoral. Pigs are just as smart, cute, and playful as dogs and dolphins.
Nobody would consider eating those animals. So why are pigs any different?
\end{quote}

That passage is trying to convince people not to eat pork, and it does so on the basis of analogy:
pigs are just like other animals we would never eat--dogs and dolphins.

Analogical arguments all share the same basic structure. We can lay out this form schematically
as follows:

\begin{quote}
a1, a2, \dots an, and c all have P1, P2, \dots Pk \\
\underline{a1, a2, \dots an all have Q} \\
c has Q \\
\end{quote}

This is an abstract schema, and it's going to take some getting used to, but it represents the form
of analogical reasoning succinctly and clearly. Arguments from analogy have two premises and a
conclusion. The first premise establishes an analogy. The analogy is between some thing, marked
`c' in the schema, and some number of other things, marked `a1', `a2', and so on in the schema.
We can refer to these as the ``analogues''. They're the things that are similar, analogous to c. This
schema is meant to cover every possible argument from analogy, so we do not specify a particular
number of analogues; the last one on the list is marked `an', where `n' is a variable standing for
any number whatsoever. There may be only one analogue; there may be a hundred. What's
important is that the analogues are similar to the thing designated by `c'. What makes different
things similar? They have stuff in common; they share properties. Those properties--the
similarities between the analogues and c--are marked `P1', `P2', and so on in the diagram. Again,
we don't specify a particular number of properties shared: the last is marked `Pk', where `k' is just
another variable (we don't use `n' again, because the number of analogues and the number of
properties can of course be different). This is because our schema is generic: every argument from
analogy fits into the framework; there may be any number of properties involved in any particular
argument. Anyway, the first premise establishes the analogy: c and the analogues are similar
because they have various things in common--P1, P2, P3, \dots Pk.

Notice that `c' is missing from the second premise. The second premise only concerns the
analogues: it says that they have some property in common, designated `Q' to highlight the fact
that it's not among the properties listed in the first premise. It's a separate property. It's the very
property we're trying to establish, in the conclusion, that c has (`c' is for conclusion). The thinking
is something like this: c and the analogues are similar in so many ways (first premise); the
analogues have this additional thing in common (Q in the second premise); so, c is probably like
that, too (conclusion: c has Q).

It will be helpful to apply these abstract considerations to concrete examples. We have two in hand.
The first argument, predicting that I would like The Wolf of Wall Street, fits the pattern. Here's the
argument again, for reference:

\begin{quote}The Wolf of Wall Street is directed by Martin Scorsese, and it stars Leonardo DiCaprio.
Those two have collaborated several times in the past, on Gangs of New York, The Aviator,
The Departed, and Shutter Island. I liked each of those movies, so I predict that I will like
The Wolf of Wall Street.\end{quote}

The conclusion is something like `I will like The Wolf of Wall Street'. Putting it that way, and
looking at the general form of the conclusion of analogical arguments (c has Q), it's tempting to
say that `c' designates me, while the property Q is something like `liking The Wolf of Wall Street'.
But that's not right. The thing that `c' designates has to be involved in the analogy in the first
premise; it has to be the thing that's similar to the analogues. The analogy that this argument hinges
on is between the various movies. It's not I that `c' corresponds to; it's the movie we're making
the prediction about. The Wolf of Wall Street is what `c' picks out. What property are we predicting
it will have? Something like `liked by me'. The analogues, the a's in the schema, are the other
movies: Gangs of New York, The Aviator, The Departed, and Shutter Island. (In this example, n is
4; the movies are a1, a2, a3, and a4.) These we know have the property Q (liked by me): I had
already seen and liked these movies. That's the second premise: that the analogues have Q. Finally,
the first premise, which establishes the analogy among all the movies. What do they have in
common? They were all directed by Martin Scorsese, and they all starred Leonardo DiCaprio.
Those are the P's--the properties they all share. P1 is `directed by Scorsese'; P2 is `stars DiCaprio'.

The second argument we considered, about eating pork, also fits the pattern. Here it is again, for
reference:

\begin{quote}Eating pork is immoral. Pigs are just as smart, cute, and playful as dogs and dolphins.
Nobody would consider eating those animals. So why are pigs any different?
\end{quote}

Again, looking at the conclusion--`Eating pork is immoral'--and looking at the general form of
conclusions for analogical arguments--`c has Q'--it's tempting to just read off from the syntax of
the sentence that `c' stands for `eating pork' and Q for `is immoral'. But that's not right. Focus on
the analogy: what things are being compared to one another? It's the animals: pigs, dogs, and
dolphins; those are our a's and c. To determine which one is picked out by `c', we ask which
animal is involved in the conclusion. It's pigs; they are picked out by `c'. So we have to paraphrase
our conclusion so that it fits the form `c has Q', where `c' stands for pigs. Something like `Pigs
shouldn't be eaten' would work. So Q is the property `shouldn't be eaten'. The analogues are dogs
and dolphins. They clearly have the property: as the argument notes, (most) everybody agrees they
shouldn't be eaten. This is the second premise. And the first establishes the analogy. What do pigs
have in common with dogs and dolphins? They're smart, cute, and playful. P1 = `is smart'; P2 =
`is cute'; and P3 = `is playful'.

\subsubsection{The Evaluation of Analogical Arguments}
Unlike in the case of deduction, we will not have to learn special techniques to use when evaluating
these sorts of arguments. It's something we already know how to do, something we typically do
automatically and unreflectively. The purpose of this section, then, is not to learn a new skill, but
rather subject a practice we already know how to engage in to critical scrutiny. We evaluate
analogical arguments all the time without thinking about how we do it. We want to achieve a
metacognitive perspective on the practice of evaluating arguments from analogy; we want to think
about a type of thinking that we typically engage in without much conscious deliberation. We want
to identify the criteria that we rely on to evaluate analogical reasoning--criteria that we apply
without necessarily realizing that we're applying them. Achieving such metacognitive awareness
is useful insofar as it makes us more self-aware, critical, and therefore effective reasoners.

Analogical arguments are inductive arguments. They give us reasons that are supposed to make
their conclusions more probable. How probable, exactly? That's very hard to say. How probable
was it that I would like The Wolf of Wall Street given that I had liked the other four
Scorsese/DiCaprio collaborations? I don't know. How probable is it that it's wrong to eat pork
given that it's wrong to eat dogs and dolphins? I really don't know. It's hard to imagine how you
would even begin to answer that question.

As we mentioned, while it's often impossible to evaluate inductive arguments by giving a precise
probability of its conclusion, it is possible to make relative judgments about strength and
weakness. Recall, new information can change the probability of the conclusion of an inductive
argument. We can make relative judgments of like this: if we add this new information as a
premise, the new argument is stronger/weaker than the old argument; that is, the new information
makes the conclusion more/less likely.

It is these types of relative judgments that we make when we evaluate analogical reasoning. We
compare different arguments--with the difference being new information in the form of an added
premise, or a different conclusion supported by the same premises--and judge one to be stronger
or weaker than the other. Subjecting this practice to critical scrutiny, we can identify six criteria
that we use to make such judgments.

We're going to be making relative judgments, so we need a baseline argument against which to
compare others. Here is such an argument:

\begin{quote}Alice has taken four Philosophy courses during her time in college. She got an A in all
four. She has signed up to take another Philosophy course this semester. I predict she will
get an A in that course, too.
\end{quote}

This is a simple argument from analogy, in which the future is predicted based on past experience.
It fits the schema for analogical arguments: the new course she has signed up for is designated by
`c'; the property we're predicting it has (Q) is that it is a course Alice will get an A in; the analogues
are the four previous courses she's taken; what they have in common with the new course (P 1) is
that they are also Philosophy classes; and they all have the property Q--Sally got an A in each.

Anyway, how strong is the baseline argument? How probable is its conclusion in light of its
premises? I have no idea. It doesn't matter. We're now going to consider tweaks to the argument,
and the effect that those will have on the probability of the conclusion. That is, we're going to
consider slightly different arguments, with new information added to the original premises or
changes to the prediction based on them, and ask whether these altered new arguments are stronger
or weaker than the baseline argument. This will reveal the six criteria that we use to make such
judgments. We'll consider one criterion at a time.

\paragraph{Number of Analogues}
Suppose we alter the original argument by changing the number of prior Philosophy courses Alice
had taken. Instead of Alice having taken four philosophy courses before, we'll now suppose she
has taken 14. We'll keep everything else about the argument the same: she got an A in all of them,
and we're predicting she'll get an A in the new one. Are we more or less confident in the
conclusion--the prediction of an A--with the altered premise? Is this new argument stronger or
weaker than the baseline argument?

It's stronger! We've got Alice getting an A 14 times in a row instead of only four. That clearly
makes the conclusion more probable. (How much more? Again, it doesn't matter.)

What we did in this case is add more analogues. This reveals a general rule: other things being
equal, the more analogues in an analogical argument, the stronger the argument (and conversely,
the fewer analogues, the weaker). The number of analogues is one of the criteria we use to evaluate
arguments from analogy.

\paragraph{Variety of Analogues}

You'll notice that the original argument doesn't give us much information about the four courses
Alice succeeded in previously and the new course she's about to take. All we know is that they're
all Philosophy courses. Suppose we tweak things. We're still in the dark about the new course
Alice is about to take, but we know a bit more about the other four: one was a course in Ancient
Greek Philosophy; one was a course on Contemporary Ethical Theories; one was a course in
Formal Logic; and the last one was a course in the Philosophy of Mind. Given this new
information, are we more or less confident that she will succeed in the new course, whose topic is
unknown to us? Is the argument stronger or weaker than the baseline argument?

It is stronger. We don't know what kind of Philosophy course Alice is about to take, but this new
information gives us an indication that it doesn't really matter. She was able to succeed in a wide
variety of courses, from Mind to Logic, from Ancient Greek to Contemporary Ethics. This is
evidence that Alice is good at Philosophy generally, so that no matter what kind of course she's
about to take, she'll probably do well in it.

Again, this points to a general principle about how we evaluate analogical arguments: other things
being equal, the more variety there is among the analogues, the stronger the argument (and
conversely, the less variety, the weaker).

\paragraph{Number of Similarities}

In the baseline argument, the only thing the four previous courses and the new course have in
common is that they're Philosophy classes. Suppose we change that. Our newly tweaked argument
predicts that Alice will get an A in the new course, which, like the four she succeeded in before,
is cross-listed in the Department of Religious Studies and covers topics in the Philosophy of
Religion. Given this new information--that the new course and the four older courses were similar
in ways we weren't aware of--are we more or less confident in the prediction that Alice will get
another A? Is the argument stronger or weaker than the baseline argument?

Again, it is stronger. Unlike the last example, this tweak gives us new information both about the
four previous courses and the new one. The upshot of that information is that they're more similar
than we knew; that is, they have more properties in common. To P1 = `is a Philosophy course' we
can add P2 = `is cross-listed with Religious Studies' and P3 = `covers topics in Philosophy of
Religion'. The more properties things have in common, the stronger the analogy between them.
The stronger the analogy, the stronger the argument based on that analogy. We now know not just
that Alice did well in not just in Philosophy classes--but specifically in classes covering the
Philosophy of Religion; and we know that the new class she's taking is also a Philosophy of
Religion class. I'm much more confident predicting she'll do well again than I was when all I knew
was that all the classes were Philosophy; the new one could've been in a different topic that she
wouldn't have liked.

General principle: other things being equal, the more properties involved in the analogy--the more
similarities between the item in the conclusion and the analogues--the stronger the argument (and
conversely, the fewer properties, the weaker).

\paragraph{Number of Differences}

An argument from analogy is built on the foundation of the similarities between the analogues and
the item in the conclusion--the analogy. Anything that weakens that foundation weakens the
argument. So, to the extent that there are differences among those items, the argument is weaker.

Suppose we add new information to our baseline argument: the four Philosophy courses Alice did
well in before were all courses in the Philosophy of Mind; the new course is about the history of
Ancient Greek Philosophy. Given this new information, are we more or less confident that she will
succeed in the new course? Is the argument stronger or weaker than the baseline argument?
Clearly, the argument is weaker. The new course is on a completely different topic than the other
ones. She did well in four straight Philosophy of Mind courses, but Ancient Greek Philosophy is
quite different. I'm less confident that she'll get an A than I was before.

If I add more differences, the argument gets even weaker. Supposing the four Philosophy of Mind
courses were all taught by the same professor (the person in the department whose expertise is in
that area), but the Ancient Greek Philosophy course is taught by someone different (the
department's specialist in that topic). Different subject matter, different teachers: I'm even less
optimistic about Alice's continued success.

Generally speaking, other things being equal, the more differences there are between the analogues
and the item in the conclusion, the weaker the argument from analogy.

\paragraph{Relevance of Similarities and Differences}

Not all similarities and differences are capable of strengthening or weakening an argument from
analogy, however. Suppose we tweak the original argument by adding the new information that
the new course and the four previous courses all have their weekly meetings in the same campus
building. This is an additional property that the courses have in common, which, as we just saw,
other things being equal, should strengthen the argument. But other things are not equal in this
case. That's because it's very hard to imagine how the location of the classroom would have
anything to do with the prediction we're making--that Alice will get an A in the course. Classroom
location is simply not relevant to success in a 
course.\footnote{I'm sure someone could come up with some elaborate backstory for Alice according to which the 
location of the
class somehow makes it more likely that she will do well, but set that aside. No such story is on the table here.}
Therefore, this new information does not
strengthen the argument. Nor does it weaken it; I'm not inclined to doubt Alice will do well in
light of the information about location. It simply has no effect at all on my appraisal of her chances.

Similarly, if we tweak the original argument to add a difference between the new class and the
other four, to the effect that while all of the four older classes were in the same building, while the
new one is in a different building, there is no effect on our confidence in the conclusion. Again,
the building in which a class meets is simply not relevant to how well someone does.

Contrast these cases with the new information that the new course and the previous four are all
taught by the same professor. Now that strengthens the argument! Alice has gotten an A four times
in a row from this professor--all the more reason to expect she'll receive another one. This tidbit
strengthens the argument because the new similarity--the same person teaches all the courses--is
relevant to the prediction we're making--that Alice will do well. Who teaches a class can make a
difference to how students do--either because they're easy graders, or because they're great
teachers, or because the student and the teacher are in tune with one another, etc. Even a difference
between the analogues and the item in the conclusion, with the right kind of relevance, can
strengthen an argument. Suppose the other four philosophy classes were taught be the same
teacher, but the new one is taught by a TA--who just happens to be her boyfriend. That's a
difference, but one that makes the conclusion--that Alice will do well--more probable.

Generally speaking, careful attention must be paid to the relevance of any similarities and
differences to the property in the conclusion; the effect on strength varies.


\paragraph{Modesty/Ambition of the Conclusion}

Suppose we leave everything about the premises in the original baseline argument the same: four
Philosophy classes, an A in each, new Philosophy class. Instead of adding to that part of the
argument, we'll tweak the conclusion. Instead of predicting that Alice will get an A in the class,
we'll predict that she'll pass the course. Are we more or less confident that this prediction will
come true? Is the new, tweaked argument stronger or weaker than the baseline argument?

It's stronger. We are more confident in the prediction that Alice will pass than we are in the
prediction that she will get another A, for the simple reason that it's much easier to pass than it is
to get an A. That is, the prediction of passing is a much more modest prediction than the prediction
of an A.

Suppose we tweak the conclusion in the opposite direction--not more modest, but more ambitious.
Alice has gotten an A in four straight Philosophy classes, she's about to take another one, and I
predict that she will do so well that her professor will suggest that she publish her term paper in
one of the most prestigious philosophical journals and that she will be offered a three-year research
fellowship at the Institute for Advanced Study at Princeton University. That's a bold prediction!
Meaning, of course, that it's very unlikely to happen. Getting an A is one thing; getting an
invitation to be a visiting scholar at one of the most prestigious academic institutions in the world
is quite another. The argument with this ambitious conclusion is weaker than the baseline
argument.

General principle: the more modest the argument's conclusion, the stronger the argument; the more
ambitious, the weaker.

\subsubsection{Refutation by Analogy}
We can use arguments from analogy for a specific logical task: refuting someone else's argument,
showing that it's bad. Recall the case of deductive arguments. To refute those--to show that they
are bad, i.e., invalid--we had to produce a counterexample--a new argument with the same logical
form as the original that was obviously invalid, in that its premises were in fact true and its
conclusion in fact false. We can use a similar procedure to refute inductive arguments. Of course,
the standard of evaluation is different for induction: we don't judge them according to the black
and white standard of validity. And as a result, our judgments have less to do with form than with
content. Nevertheless, refutation along similar lines is possible, and analogies are the key to the
technique.

To refute an inductive argument, we produce a new argument that's obviously bad--just as we did
in the case of deduction. We don't have a precise notion of logical form for inductive arguments,
so we can't demand that the refuting argument have the same form as the original; rather, we want
the new argument to have an analogous form to the original. The stronger the analogy between
the refuting and refuted arguments, the more decisive the refutation. We cannot produce the kind
of knock-down refutations that were possible in the case of deductive arguments, where the
standard of evaluation--validity--does not admit of degrees of goodness or badness, but the
technique can be quite effective.

Consider the following:

\begin{quote}``Duck Dynasty'' star and Duck Commander CEO Willie Robertson said he supports Trump
because both of them have been successful businessmen and stars of reality TV shows. \\
\end{quote}

\begin{quote}By that logic, does that mean Hugh Hefner's success with ``Playboy'' and his occasional
appearances on ``Bad Girls Club'' warrant him as a worthy president? Actually, I'd still be
more likely to vote for Hefner than 
Trump.\footnote{Austin Faulds, ``Weird celebrity endorsements fit for weird election,'' Indiana Daily Student, 10/12/16,
http://www.idsnews.com/article/2016/10/weird-celebrity-endorsements-for-weird-election.}
\end{quote}


The author is refuting the argument of Willie Robertson, the ``Duck Dynasty'' star. Robertson's
argument is something like this: Trump is a successful businessman and reality TV star; therefore,
he would be a good president. To refute this, the author produces an analogous argument--Hugh
Hefner is a successful businessman and reality TV star; therefore, Hugh Hefner would make a
good president--that he regards as obviously bad. What makes it obviously bad is that it has a
conclusion that nobody would agree with: Hugh Hefner would make a good president. That's how
these refutations work. They attempt to demonstrate that the original argument is lousy by showing
that you can use the same or very similar reasoning to arrive at an absurd conclusion.

Here's another example, from a group called ``Iowans for Public Education''. Next to a picture of
an apparently well-to-do lady is the following text:

\begin{quote}
``My husband and I have decided the local parks just aren't good enough for our kids. We'd
rather use the country club, and we are hoping state tax dollars will pay for it. We are
advocating for Park Savings Accounts, or PSAs. We promise to no longer use the local
parks. To hell with anyone else or the community as a whole. We want our tax dollars to
be used to make the best choice for our family.'' \\
\end{quote}

\begin{quote}
Sound ridiculous? Tell your legislator to vote NO on Education Savings Accounts (ESAs),
aka school vouchers.
\end{quote}

The argument that Iowans for Public Education put in the mouth of the lady on the poster is meant
to refute reasoning used by advocates for ``school choice'', who say that they ought to have the
right to opt out of public education and keep the tax dollars they would otherwise pay for public
schools and use it to pay to send their kids to private schools. A similar line of reasoning sounds
pretty crazy when you replace public schools with public parks and private schools with country
clubs.

Since these sorts of refutations rely on analogies, they are only as strong as the analogy between
the refuting and refuted arguments. There is room for dispute on that question. Advocates for
school vouchers might point out that schools and parks are completely different things, that schools
are much more important to the future prospects of children, and that given the importance of
education, families should have to right choose what they think is best. Or something like that.
The point is, the kinds of knock-down refutations that were possible for deductive arguments are
not possible for inductive arguments. There is always room for further debate. \\


EXERCISES \\

\begin{enumerate}
\item Show how the following arguments fit the abstract schema for arguments from analogy:

\begin{quote}
a1, a2, \dots an, and c all have P1, P2, \dots Pk \\
\underline{a1, a2, \dots an all have Q} \\
c has Q \\
\end{quote}


\begin{enumerate}
\item You should really eat at Papa Giorgio's; you'll love it. It's just like Mama DiSilvio's
and Matteo's, which I know you love: they serve old-fashioned Italian-American food,
they have a laid-back atmosphere, and the wine list is extensive.
\item George R.R. Martin deserves to rank among the greats in the fantasy literature genre.
Like C.S. Lewis and J.R.R. Tolkien before him, he has created a richly detailed world,
populated it with compelling characters, and told a tale that is not only exciting, but which
features universal and timeless themes concerning human nature.
\item Yes, African Americans are incarcerated at higher rates than whites. But blaming this
on systemic racial bias in the criminal justice system is absurd. That's like saying the NBA
is racist because there are more black players than white players, or claiming that the
medical establishment is racist because African Americans die young more often.
\end{enumerate}
\item Consider the following base-line argument:

\begin{quote}I've taken vacations to Florida six times before, and I've enjoyed each visit. I'm planning
to go to Florida again this year, and I fully expect yet another enjoyable vacation.
\end{quote}

Decide whether each of the following changes produces an argument that's weaker or stronger
than the baseline argument, and indicate which of the six criteria for evaluating analogical
arguments justifies that judgment.

\begin{enumerate}
\item All of my trips were visits to Disney World, and this one will be no different.
\item In fact, I've vacationed in Florida 60 times and enjoyed every visit.
\item I expect that I will enjoy this trip so much I will decide to move to Florida.
\item On my previous visits to Florida, I've gone to the beaches, the theme parks, the
Everglades National Park, and various cities, from Jacksonville to Key West.
\item I've always flown to Florida on Delta Airlines in the past; this time I'm going on a
United flight.
\item All of my past visits were during the winter months; this time I'm going in the summer.
\item I predict that I will find this trip more enjoyable than a visit to the dentist.
\item I've only been to Florida once before.
\item On my previous visits, I drove to Florida in my Dodge minivan, and I'm planning on
driving the van down again this time.
\item All my visits have been to Daytona Beach for the Daytona 500; same thing this time.
\item I've stayed in beachside bungalows, big fancy hotels, time-share condominiums--even
a shack out in the swamp.
\end{enumerate}
\item For each of the following passages, explicate the argument being refuted and the argument or
arguments doing the refuting.
\begin{enumerate}
\item Republicans tell us that, because at some point 40 years from now a shortfall in revenue
for Social Security is projected, we should cut benefits now. Cut them now because we
might have to cut them in the future? I've got a medium-sized tree in my yard. 40 years
from now, it may grow so large that its branches hang over my roof. Should I chop it down?
\item Opponents of gay marriage tell us that it flies in the face of a tradition going back
millennia, that marriage is between a man and a woman. There were lots of traditions that
lasted a long time: the tradition that it was OK for some people to own other people as
slaves, the tradition that women couldn't participate in the electoral process--the list goes
on. That it's traditional doesn't make it right.
\item Some people claim that their children should be exempted from getting vaccinated for
common diseases because the practice conflicts with their religious beliefs. But religion
can't be used to justify just anything. If a Satanist tried to defend himself against charges
of abusing children by claiming that such practices were a form of religious expression,
would we let him get away with it?
\end{enumerate}
\end{enumerate}
