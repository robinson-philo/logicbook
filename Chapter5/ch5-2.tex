%FROM MOD 5-2 P157
%https://robjhyndman.com/hyndsight/latex-floats/
\chapter{Causal reasoning}
When I strike a match it will produce a flame. It is natural to take the striking of
the match as the cause that produces the effect of a flame. But what if the
matchbook is wet? Or what if I happen to be in a vacuum in which there is no
oxygen (such as in outer space)? If either of those things is the case, then the
striking of the match will not produce a flame. So it isn't simply the striking of
the match that produces the flame, but a combination of the striking of the
match together with a number of other conditions that must be in place in order
for the striking of the match to create a flame. Which of those conditions we call
the ``cause'' depends in part on the context. Suppose that I'm in outer space
striking a match (suppose I'm wearing a space suit that supplies me with oxygen
but that I'm striking the match in space, where there is no oxygen). I
continuously strike it but no flame appears (of course). But then someone (also
in a space suit) brings out a can of compressed oxygen that they spray on the
match while I strike it. All of a sudden a flame is produced. In this context, it
looks like it is the spraying of oxygen that causes flame, not the striking of the
match. Just as in the case of the striking of the match, any cause is more
complex than just a simple event that produces some other event. Rather, there
are always multiple conditions that must be in place for any cause to occur.
These conditions are called \emph{background conditions}. That said, we often take
for granted the background conditions in normal contexts and just refer to one
particular event as the cause. Thus, we call the striking of the match the cause
of the flame. We don't go on to specify all the other conditions that conspired
to create the flame (such as the presence of oxygen and the absence of water).
But this is more for convenience than correctness. For just about any cause,
there are a number of conditions that must be in place in order for the effect to
occur. These are called necessary conditions (recall the discussion of necessary
and sufficient conditions from chapter 2, section 2.7). For example, a necessary
condition of the match lighting is that there is oxygen present. A necessary
condition of a car running is that there is gas in the tank. We can use necessary
conditions to diagnose what has gone wrong in cases of malfunction. That is,
we can consider each condition in turn in order to determine what caused the
malfunction. For example, if the match doesn't light, we can check to see
whether the matches are wet. If we find that the matches are wet then we can
explain the lack of the flame by saying something like, ``dropping the matches in
the water caused the matches not to light.'' In contrast, a sufficient condition is
one which if present will always bring about the effect. For example, a person
being fed through an operating wood chipper is sufficient for causing that
person's death (as was the fate of Steve Buscemi's character in the movie Fargo).

Because the natural world functions in accordance with natural laws (such as the
laws of physics), causes can be generalized. For example, any object near the
surface of the earth will fall towards the earth at 9.8 m/s2 unless impeded by
some contrary force (such as the propulsion of a rocket). This generalization
applies to apples, rocks, people, wood chippers and every other object. Such
causal generalizations are often parts of explanations. For example, we can
explain why the airplane crashed to the ground by citing the causal
generalization that all unsupported objects fall to the ground and by noting that
the airplane had lost any method of propelling itself because the engines had
died. So we invoke the causal generalization in explaining why the airplane
crashed. Causal generalizations have a particular form:

\begin{quote}
For any x, if x has the feature(s) F, then x has the feature G
\end{quote}

For example:

\begin{quote}
For any human, if that human has been fed through an operating wood
chipper, then that human is dead.
\end{quote}

\begin{quote}
For any engine, if that engine has no fuel, then that engine will
not operate.
\end{quote}

\begin{quote}
For any object near the surface of the earth, if that object is unsupported
and not impeded by some contrary force, then that object will fall
towards the earth at 9.8 m/s2.
\end{quote}

Being able to determine when causal generalizations are true is an important
part of becoming a critical thinker. Since in both scientific and every day
contexts we rely on causal generalizations in explaining and understanding our
world, the ability to assess when a causal generalization is true is an important
skill. For example, suppose that we are trying to figure out what causes our
dog, Charlie, to have seizures. To simplify, let's suppose that we have a set of
potential candidates for what causes his seizures. It could be either:

\begin{enumerate}
\item eating human food,
\item the shampoo we use to wash him,
\item his flea treatment,
\item not eating at regular intervals,
\end{enumerate}

or some combination of these things. Suppose we keep a log of when these
things occur each day and when his seizures (S) occur. In the table below, I will
represent the absence of the feature by a negation. So in the table below, ``$\sim$A''
represents that Charlie did not eat human food on that day; ``$\sim$B'' represents
that he did not get a bath and shampoo that day; ``$\sim$S'' represents that he did
not have a seizure that day. In contrast, ``B'' represents that he did have a bath
and shampoo, whereas ``C'' represents that he was given a flea treatment that
day. Here is how the log looks:


\begin{table}[htp]
\begin{tabular}{|l|l|l|l|l|l|}
\hline
Day 1 & $\sim$A & B       & C       & D       & S       \\
\hline
Day 2 & A       & $\sim$B & C       & D       & $\sim$S \\
\hline
Day 3 & A       & B       & $\sim$C & D       & $\sim$S \\
\hline
Day 4 & A       & B       & C       & $\sim$D & S       \\
\hline
Day 5 & A       & B       & $\sim$C & D       & $\sim$S \\
\hline
Day 6 & A       & $\sim$B & C       & D       & $\sim$S \\
\hline
\end{tabular}
\end{table}


How can we use this information to determine what might be causing Charlie to
have seizures? The first thing we'd want to know is what feature is present every
time he has a seizure. This would be a necessary (but not sufficient) condition.
And that can tell us something important about the cause. The necessary
condition test says that any candidate feature (here A, B, C, or D) that is absent
when the target feature (S) is present is eliminated as a possible necessary
condition of S.\footnote{This discussion draws heavily on chapter 10, pp. 220-224 of 
Sinnott-Armstrong and Fogelin's
Understanding Arguments, 9th edition (Cengage Learning).}
In the table above, A is absent when S is present, so A can't be a
necessary condition (i.e., day 1). D is also absent when S is present (day 4) so D
can't be a necessary condition either. In contrast, B is never absent when S is
present--that is every time S is present, B is also present. That means B is a
necessary condition, based on the data that we have gathered so far. The same
applies to C since it is never absent when S is present. Notice that there are
times when both B and C are absent, but on those days the target feature (S) is
absent as well, so it doesn't matter.

The next thing we'd want to know is which feature is such that every time it is
present, Charlie has a seizure. The test that is relevant to determining this is
called the sufficient condition test. The sufficient condition test says that any
candidate that is present when the target feature (S) is absent is eliminated as a
possible sufficient condition of S. In the table above, we can see that no one
candidate feature is a sufficient condition for causing the seizures since for each
candidate (A, B, C, D) there is a case (i.e. day) where it is present but that no
seizure occurred. Although no one feature is sufficient for causing the seizures
(according to the data we have gathered so far), it is still possible that certain
features are jointly sufficient. Two candidate features are jointly sufficient for a
target feature if and only if there is no case in which both candidates are present
and yet the target is absent. Applying this test, we can see that B and C are
jointly sufficient for the target feature since any time both are present, the target
feature is always present. Thus, from the data we have gathered so far, we can
say that the likely cause of Charlie's seizures are when we both give him a bath
and then follow that bath up with a flea treatment. Every time those two things
occur, he has a seizure (sufficient condition); and every time he has a seizure,
those two things occur (necessary condition). Thus, the data gathered so far
supports the following causal conditional:

\begin{quote}Any time Charlie is given a shampoo bath and a flea treatment, he has a
seizure.
\end{quote}

Although in the above case, the necessary and sufficient conditions were the
same, this needn't always be the case. Sometimes sufficient conditions are not
necessary conditions. For example, being fed through a wood chipper is a
sufficient condition for death, but it certainly isn't necessary! (Lot's of people die
without being fed through a wood chipper, so it can't be a necessary condition
of dying.) In any case, determining necessary and sufficient conditions is a key
part of determining a cause.

When analyzing data to find a cause it is important that we rigorously test each
candidate. Here is an example to illustrate rigorous testing. Suppose that on
every day we collected data about Charlie he ate human food but that on none
of the days was he given a bath and shampoo, as the table below indicates.


\begin{table}[htp]
\begin{tabular}{|l|l|l|l|l|l|}
\hline
Day 1 & A & $\sim$B & C       & D       & $\sim$S \\
\hline
Day 2 & A & $\sim$B & C       & D       & $\sim$S \\
\hline
Day 3 & A & $\sim$B & $\sim$C & D       & $\sim$S \\
\hline
Day 4 & A & $\sim$B & C       & $\sim$D & S       \\
\hline
Day 5 & A & $\sim$B & $\sim$C & D       & $\sim$S \\
\hline
Day 6 & A & $\sim$B & C       & D       & S      \\
\hline
\end{tabular}
\end{table}


Given this data, A trivially passes the necessary condition test since it is always
present (thus, there can never be a case where A is absent when S is present).
However, in order to rigorously test A as a necessary condition, we have to look
for cases in which A is not present and then see if our target condition S is
present. We have rigorously tested A as a necessary condition only if we have
collected data in which A was not present. Otherwise, we don't really know
whether A is a necessary condition. Similarly, B trivially passes the sufficient
condition test since it is never present (thus, there can never be a case where B
is present but S is absent). However, in order to rigorously test B as a sufficient
condition, we have to look for cases in which B is present and then see if our
target condition S is absent. We have rigorously tested B as a sufficient
condition only if we have collected data in which B is present. Otherwise, we
don't really know whether B is a sufficient condition or not.

In rigorous testing, we are actively looking for (or trying to create) situations in
which a candidate feature fails one of the tests. That is why when rigorously
testing a candidate for the necessary condition test, we must seek out cases in
which the candidate is not present, whereas when rigorously testing a candidate
for the sufficient condition test, we must seek out cases in which the candidate is
present. In the example above, A is not rigorously tested as a necessary
condition and B is not rigorously tested as a sufficient condition. If we are
interested in finding a cause, we should always rigorously test each candidate.
This means that we should always have a mix of different situations where the
candidates and targets are sometimes present and sometimes absent. \\

\newpage
EXERCISES \\

Determine which of the candidates (A, B, C, D) in the
following examples pass the necessary condition test or the sufficient
condition test relative to the target (G). In addition, note whether there
are any candidates that aren't rigorously tested as either necessary or
sufficient conditions.

\begin{table}[htp]
\begin{tabular}{|l|l|l|l|l|l|}
\hline
Case 1 & A       & B       & $\sim$C & D & $\sim$G \\
\hline
Case 2 & $\sim$A & B       & C       & D & G       \\
\hline
Case 3 & A       & $\sim$B & C       & D & G      \\
\hline
\end{tabular}
\end{table}

\begin{table}[htp]
\begin{tabular}{|l|l|l|l|l|l|}
\hline
Case 1 & A       & B       & C       & D       & G       \\
\hline
Case 2 & $\sim$A & B       & $\sim$C & D       & $\sim$G \\
\hline
Case 3 & A       & $\sim$B & C       & $\sim$D & G  \\    
\hline
\end{tabular}
\end{table}

\begin{table}[htp]
\begin{tabular}{|l|l|l|l|l|l|}
\hline
Case 1 & A       & B       & C & D & G \\
\hline
Case 2 & $\sim$A & B       & C & D & G \\
\hline
Case 3 & A       & $\sim$B & C & D & G \\
\hline
\end{tabular}
\end{table}

\begin{table}[htp]
\begin{tabular}{|l|l|l|l|l|l|}
\hline
Case 1 & A       & B & C & D       & $\sim$G \\
\hline
Case 2 & $\sim$A & B & C & D       & G       \\
\hline
Case 3 & A       & B & C & $\sim$D & G   \\   
\hline
\end{tabular}
\end{table}

\begin{table}[htp]
\begin{tabular}{|l|l|l|l|l|l|}
\hline
Case 1 & A       & B       & $\sim$C & D       & $\sim$G \\
\hline
Case 2 & $\sim$A & B       & C       & D       & G       \\
\hline
Case 3 & A       & $\sim$B & $\sim$C & $\sim$D & $\sim$G \\
\hline
\end{tabular}
\end{table}

\begin{table}[htp]
\begin{tabular}{|l|l|l|l|l|l|}
\hline
Case 1 & A       & B       & C       & D       & $\sim$G \\
Case 2 & $\sim$A & B       & C       & $\sim$D & $\sim$G \\
Case 3 & A       & $\sim$B & $\sim$C & D       & G      \\
\hline
\end{tabular}
\end{table}

\newpage
\begin{itemize}
\item For each of the following correlations, use your background
knowledge to determine whether A causes B, B causes A, a common
cause C is the cause of both A and B, or the correlations is accidental.
\end{itemize}

\begin{enumerate}
\item There is a positive correlation between U.S. spending on science,
space, and technology (A) and suicides by hanging, strangulation, and
suffocation (B).
\item There is a positive correlation between our dog Charlie's weight (A)
and the amount of time we spend away from home (B). That is, the
more time we spend away from home, the heavier Charlie gets (and
the more we are at home, the lighter Charlie is.
\item The height of the tree in our front yard (A) positively correlates with
the height of the shrub in our backyard (B).
\item There is a negative correlation between the number of suicide
bombings in the U.S. (A) and the number of hairs on a particular U.S
President's head (B).
\item There is a high positive correlation between the number of fire
engines in a particular borough of New York Cite (A) and the number
of fires that occur there (B).
\item At one point in history, there was a negative correlation between the
number of mules in the state (A) and the salaries paid to professors at
the state university (B). That is, the more mules, the lower the
professors' salaries.
\item There is a strong positive correlation between the number of traffic
accidents on a particular highway (A) and the number of billboards
featuring scantily-clad models (B).
\item The girth of an adult's waist (A) is negatively correlated with the height
of their vertical leap (B).
\item Olympic marathon times (A) are positively correlated with the
temperature during the marathon (B). That is, the more time it takes
an Olympic marathoner to complete the race, the higher the
temperature.
\item The number gray hairs on an individual's head (A) is positively
correlated with the number of children or grandchildren they have (B).
\end{enumerate}
