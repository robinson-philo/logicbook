%mod3-1 pg 10

\chapter{Deductive and Inductive Arguments}
As we noted earlier, there are different logics--different approaches to distinguishing good
arguments from bad ones. One of the reasons we need different logics is that there are different
kinds of arguments. In this section, we distinguish two types: deductive and inductive arguments.

\subsection{Deductive Arguments}
First, deductive arguments. These are distinguished by their aim: a deductive argument attempts
to provide premises that guarantee, necessitate its conclusion. Success for a deductive argument,
then, does not come in degrees: either the premises do in fact guarantee the conclusion, in which
case the argument is a good, successful one, or they don't, in which case it fails. Evaluation of
deductive arguments is a black-and-white, yes-or-no affair; there is no middle ground.
We have a special term for a successful deductive argument: we call it valid. Validity is a central
concept in the study of logic. It's so important, we're going to define it three times. Each of these
three definitions is equivalent to the others; they are just three different ways of saying the same
thing:

\begin{quote}
An argument is valid just in case \dots \\
(i) its premises guarantee its conclusion; i.e., \\
(ii) if its premises are true, then its conclusion must also be true; i.e., \\
(iii) it is impossible for its premises to be true and its conclusion false. \\
\end{quote}

Here's an example of a valid deductive argument:

\begin{quote}
All humans are mortal. \\
\underline{Socrates is a human.} \\
Socrates is mortal.\\
\end{quote}

This argument is valid because the premises do in fact guarantee the conclusion: if they're true (as
a matter of fact, they are), then the conclusion must be true; it's impossible for the premises to be
true and the conclusion false.

Here's a surprising fact about validity: what makes a deductive argument valid has nothing to do
with its content; rather, validity is determined by the argument's form. That is to say, what makes
our Socrates argument valid is not that it says a bunch of accurate things about Socrates, humanity,
and mortality. The content doesn't make a difference. Instead, it's the form that matters--the
pattern that the argument exhibits.

Later, when undertake a more detailed study of deductive logic, we will give a precise definition
of logical form.\footnote{Definitions, actually. We'll study two different deductive logics, each with its own definition of form.}
For now, we'll use this rough gloss: the form of an argument is what's left over
when you strip away all the non-logical terms and replace them with 
blanks.\footnote{What counts as a ``logical term,'' you're wondering? Unhelpful answer: it depends on the logic; different logics count
different terms as logical. Again, this is just a rough gloss. We don't need precision just yet, but we'll get it eventually.}

Here's what that looks like for our Socrates argument:

\begin{quote}
All A are B. \\
\underline{x is A.} \\
x is B. \\
\end{quote}

The letter are the blanks: they're placeholders, variables. As a matter of convention, we're using
capital letters to stand for groups of things (humans, mortals) and lower case letters to stand for
individual things (Socrates).

The Socrates argument is a good, valid argument because it exhibits this good, valid form. Our
third way of wording the definition of validity helps us see why this is a valid form: it's impossible
for the premises to be true and the conclusion false, in that it's impossible to plug in terms for A,
B, and x in such a way that the premises come out true and the conclusion comes out false.
A consequence of the fact that validity is determined entirely by an argument's form is that, given
a valid form, every single argument that has that form will be valid. So any argument that has the
same form as our Socrates argument will be valid; that is, we can pick things at random to stick in
for A, B, and x, and we're guaranteed to get a valid argument. Here's a silly example:

\begin{quote}
All apples are bananas. \\
\underline{Donald Trump is an apple.} \\
Donald Trump is a banana. \\
\end{quote}

This argument has the same form as the Socrates argument: we simply replaced A with `apples',
B with `bananas', and x with `Donald Trump'. That means it's a valid argument. That's a strange
thing to say, since the argument is just silly--but it's the form that matters, not the content. Our
second way of wording the definition of validity can help us here. The standard for validity is this:
IF the premises are true, then the conclusion must be. That's a big `IF'. In this case, as a matter of
fact, the premises are not true (they're silly, plainly false). However, IF they were true--if in fact
apples were a type of banana and Donald Trump were an apple--then the conclusion would be
unavoidable: Trump would have to be a banana. The premises aren't true, but if they were, the
conclusion would have to be--that's validity.

So it turns out that the actual truth or falsehood of the propositions in a valid argument are
completely irrelevant to its validity. The Socrates argument has all true propositions and it's valid;
the Donald Trump argument has all false propositions, but it's valid, too. They're both valid
because they have a valid form; the truth/falsity of their propositions don't make any difference.
This means that a valid argument can have propositions with almost any combination of truthvalues: 
some true premises, some false ones, a true or false conclusion. One can fiddle around with
the Socrates' argument's form, plugging different things in for A, B, and x, and see that this is so.
For example, plug in `ants' for A, `bugs' for B, and Beyoncé for x: you get one true premise (All
ants are bugs), one false one (Beyonc\'{e} is an ant), and a false conclusion (Beyonc\'{e} is a bug). Plug
in other things and you can get any other combination of truth-values.

Any combination, that is, but one: you'll never get true premises and a false conclusion. That's
because the Socrates' argument's form is a valid one; by definition, it's impossible to generate true
premises and a false conclusion in that case.

This irrelevance of truth-value to judgments about validity means that those judgments are immune
to revision. That is, once we decide whether an argument is valid or not, that decision cannot be
changed by the discovery of new information. New information might change our judgment about
whether a particular proposition in our argument is true or false, but that can't change our judgment
about validity. Validity is determined by the argument's form, and new information can't change
the form of an argument. The Socrates argument is valid because it has a valid form. Suppose we
discovered, say, that as a matter of fact Socrates wasn't a human being at all, but rather an alien
from outer space who got a kick out of harassing random people on the streets of ancient Athens.
That information would change the argument's second premise--Socrates is human--from a truth
to a falsehood. But it wouldn't make the argument invalid. The form is still the same, and it's a
valid one.

It's time to face up to an awkward consequence of our definition of validity. Remember, logic is
about evaluating arguments--saying whether they're good or bad. We've said that for deductive
arguments, the standard for goodness is validity: the good deductive arguments are the valid ones.
Here's where the awkwardness comes in: because validity is determined by form, it's possible to
generate valid arguments that are nevertheless completely ridiculous-sounding on their face.
Remember, the Donald Trump argument--where we concluded that he's a banana--is valid. In
other words, we're saying that the Trump argument is good; it's valid, so it gets the logical thumbsup. 
But that's nuts! The Trump argument is obviously bad, in some sense of `bad', right? It's a
collection of silly, nonsensical claims.

We need a new concept to specify what's wrong with the Trump argument. That concept is
soundness. This is a higher standard of argument-goodness than validity; in order to meet it, an
argument must satisfy two conditions.

\begin{quote}
An argument is sound just in case (i) it's valid, AND (ii) its premises are in fact 
true.\footnote{What about the conclusion? Does it have to be true? Yes: remember, for valid arguments, if the premises are true,
the conclusion has to be. Sound arguments are valid, so it goes without saying that the conclusion is true, provided
that the premises are.}
\end{quote}

The Trump argument, while valid, is not sound, because it fails to satisfy the second condition: its
premises are both false. The Socrates argument, however, which is valid and contains nothing but
truths (Socrates was not in fact an alien), is sound.

The question now naturally arises: if soundness is a higher standard of argument-goodness than
validity, why didn't we say that in the first place? Why so much emphasis on validity? The answer
is this: we're doing logic here, and as logicians, we have no special insight into the soundness of
arguments. Or rather, we should say that as logicians, we have only partial expertise on the
question of soundness. Logic can tell us whether or not an argument is valid, but it cannot tell us
whether or not it is sound. Logic has no special insight into the second condition for soundness,
the actual truth-values of premises. To take an example from the silly Trump argument, suppose
you weren't sure about the truth of the first premise, which claims that all apples are bananas (you
have very little experience with fruit, apparently). How would you go about determining whether
that claim was true or false? Whom would you ask? Well, this is a pretty easy one, so you could
ask pretty much anybody, but the point is this: if you weren't sure about the relationship between
apples and bananas, you wouldn't think to yourself, ``I better go find a logician to help me figure
this out.'' Propositions make claims about how things are in the world. To figure out whether
they're true or false, you need to consult experts in the relevant subject-matter. Most claims aren't
about logic, so logic is very little help in determining truth-values. Since logic can only provide
insight into the validity half of the soundness question, we focus on validity and leave soundness
to one side.

Returning to validity, then, we're now in a position to do some actual logic. Given what we know,
we can demonstrate invalidity; that is, we can prove that an invalid argument is invalid, and
therefore bad (it can't be sound, either; the first condition for soundness is validity, so if the
argument's invalid, the question of actual truth-values doesn't even come up). Here's how:

To demonstrate the invalidity of an argument, one must write a down a new argument with
the same form as the original, whose premises are in fact true and whose conclusion is in
fact false. This new argument is called a counterexample.

Let's look at an example. The following argument is invalid:

\begin{quote}
Some mammals are swimmers. \\
\underline{All whales are swimmers.} \\
All whales are mammals. \\
\end{quote}

Now, it's not really obvious that the argument is invalid. It does have one thing going for it: all the
claims it makes are true. But we know that doesn't make any difference, since validity is
determined by the argument's form, not its content. If this argument is invalid, it's invalid because
it has a bad, invalid form. This is the form:

\begin{quote}
Some A are B. \\
\underline{All C are B.} \\
All C are A. \\
\end{quote}

To prove that the original whale argument is invalid, we have to show that this form is invalid. For
a valid form, we learned, it's impossible to plug things into the blanks and get true premises and a
false conclusion; so for an invalid form, it's possible to plug things into the blanks and get that
result. That's how we generate our counterexample: we plug things in for A, B, and C so that the
premises turn out true and the conclusion turns out false. There's no real method here; you just use
your imagination to come up with an A, B, and C that give the desired 
result.\footnote{Possibly helpful hint: universal generalizations (All \dots are \dots ) are rarely true, so if you have to make one true,
as in this example, it might be good to start there; likewise, particular claims (Some \dots are \dots ) are rarely false, so
if you have to make one false--you don't in this particular example, but if you had one as a conclusion, you would--
that would be a good place to start.}

Here's a
counterexample:

\begin{quote}
Some lawyers are American citizens. \\
\underline{All members of Congress are American citizens.} \\
All members of Congress are lawyers. \\
\end{quote}

For A, we inserted `lawyers', for B we chose `American citizens', and for C, `members of
Congress'. The first premise is clearly true. The second premise is true: non-citizens aren't eligible
to be in Congress. And the conclusion is false: there are lots of people in Congress who are nonlawyers--doctors, businesspeople, etc.
That's all we need to do to prove that the original whale-argument is invalid: come up with one
counterexample, one way of filling in the blanks in its form to get true premises and a false
conclusion. We only have to prove that it's possible to get true premises and a false conclusion,
and for that, you only need one example.

What's far more difficult is to prove that a particular argument is valid. To do that, we'd have to
show that its form is such that it's impossible to generate a counterexample, to fill in the blanks to
get true premises and a false conclusion. Proving that it's possible is easy; you only need one
counterexample. Proving that it's impossible is hard; in fact, at first glance, it looks impossibly
hard! What do you do? Check all the possible ways of plugging things into the blanks, and make
sure that none of them turn out to have true premises and a false conclusion? That's nuts! There
are, literally, infinitely many ways to fill in the blanks in an argument's form. Nobody has the time
to check infinitely many potential counterexamples.

Well, take heart; it's still early. For now, we're able to do a little bit of deductive logic: given an
invalid argument, we can demonstrate that it is in fact invalid. We're not yet in the position we'd
like to be in, namely of being able to determine, for any argument whatsoever, whether it's valid
or not. Proving validity looks too hard based on what we know so far. But we'll know more later:
in chapters 3 and 4 we will study two deductive logics, and each one will give us a method of
deciding whether or not any given argument is valid. But that'll have to wait. Baby steps.

\subsubsection{Inductive Arguments}
That's all we'll say for now about deductive arguments. On to the other type of argument we're
introducing in this section: inductive arguments. These are distinguished from their deductive
cousins by their relative lack of ambition. Whereas deductive arguments aim to give premises that
guarantee/necessitate the conclusion, inductive arguments are more modest: they aim merely to
provide premises that make the conclusion more probable than it otherwise would be; they aim to
support the conclusion, but without making it unavoidable.
Here is an example of an inductive argument:

\begin{quote}
I'm telling you, you're not going die taking a plane to visit us. Airplane crashes happen far
less frequently than car crashes, for example; so you're taking a bigger risk if you drive. In
fact, plane crashes are so rare, you're far more likely to die from slipping in the bathtub.
You're not going to stop taking showers, are you?
\end{quote}

The speaker is trying to convince her visitor that he won't die in a plane crash on the way to visit
her. That's the conclusion: you won't die. This claim is supported by the others--which emphasize
how rare plane crashes are--but it is not guaranteed by them. After all, plane crashes sometimes
do happen. Instead, the premises give reasons to believe that the conclusion--you won't die--is
very probable.

Since inductive arguments have a different, more modest goal than their deductive cousins, it
would be unreasonable for us to apply the same evaluative standards to both kinds of argument.
That is, we can't use the terms `valid' and `invalid' to apply to inductive arguments. Remember,
for an argument to be valid, its premises must guarantee its conclusion. But inductive arguments
don't even try to provide a guarantee of the conclusion; technically, then, they're all invalid. But
that won't do. We need a different evaluative vocabulary to apply to inductive arguments. We will
say of inductive arguments that they are (relatively) strong or weak, depending on how probable
their conclusions are in light of their premises. One inductive argument is stronger than another
when its conclusion is more probable than the other, given their respective premises.

One consequence of this difference in evaluative standards for inductive and deductive arguments
is that for the former, unlike the latter, our evaluations are subject to revision in light of new
evidence. Recall that since the validity or invalidity of a deductive argument is determined entirely
by its form, as opposed to its content, the discovery of new information could not affect our
evaluation of those arguments. The Socrates argument remained valid, even if we discovered that
Socrates was in fact an alien. Our evaluations of inductive arguments, though, are not immune to
revision in this way. New information might make the conclusion of an inductive argument more
or less probable, and so we would have to revise our judgment accordingly, saying that the
argument is stronger or weaker. Returning to the example above about plane crashes, suppose we
were to discover that the FBI in the visitor's hometown had recently being hearing lots of ``chatter''
from terrorist groups active in the area, with strong indications that they were planning to blow up
a passenger plane. Yikes! This would affect our estimation of the probability of the conclusion of
the argument--that the visitor wasn't going to die in a crash. The probability of not dying goes
down (as the probability of dying goes up). This new information would trigger a re-evaluation of
the argument, and we would say it's now weaker. If, on the other hand, we were to learn that the
airline that flies between the visitor's and the speaker's towns had recently upgraded its entire
fleet, getting rid of all of its older planes, replacing them with newer, more reliable model, while
in addition instituting a new, more thorough and rigorous program of pre- and post-flight safety
and maintenance inspections--well, then we might revise our judgment in the other direction.

Given this information, we might judge that things are even safer for the visitor as it regards plane
travel; that is, the proposition that the visitor won't die is now even more probable than it was
before. This new information would strengthen the argument to that conclusion.

Reasonable follow-up question: how much is the argument strengthened or weakened by the new
information imagined in these scenarios? Answer: how should I know? Sorry, that's not very
helpful. But here's the point: we're talking about probabilities here; sometimes it's hard to know
what the probability of something happening really is. Sometimes it's not: if I flip a coin, I know
that the probability of it coming up tails is 0.5. But how probable is it that a particular plane from
Airline X will crash with our hypothetical visitor on board? I don't know. And how much more
probable is a disaster on the assumption of increased terrorist chatter? Again, I have no idea. All I
know is that the probability of dying on the plane goes up in that case. And in the scenario in which
Airline X has lots of new planes and security measures, the probability of a crash goes down.

Sometimes, with inductive arguments, all we can do is make relative judgments about strength
and weakness: in light of these new facts, the conclusion is more or less probable than it was before
we learned of the new facts. Sometimes, however, we can be precise about probabilities and make
absolute judgments about strength and weakness: we can say precisely how probable a conclusion
is in light of the premises supporting it. But this is a more advanced topic. We will discuss inductive
logic in chapters 5 and 6, and will go into more depth then. Until then, patience. Baby steps. \\

EXERCISES \\

1. Determine whether the following statements are true or false.

\begin{enumerate}
\item Not all valid arguments are sound.
\item An argument with a false conclusion cannot be sound.
\item An argument with true premises and a true conclusion is valid.
\item An argument with a false conclusion cannot be valid.
\end{enumerate}

\noindent
2. Demonstrate that the following argument is invalid.

\begin{quote}
Some politicians are Democrats. \\
\underline{Hillary Clinton is a politician.} \\
Hillary Clinton is a Democrat. \\
\end{quote}

The argument's form is:

\begin{quote}
Some A are B. \\
\underline{x is A.} \\
x is B. \\
\end{quote}

(where `A' and `B' stand for groups of things and `x' stands for an individual) \\


\noindent
3. Consider the following inductive argument (about a made-up person):

\noindent
Sally Johansson does all her grocery shopping at an organic food co-op. She's a huge fan
of tofu. She's really into those week-long juice cleanse thingies. And she's an active
member of PETA. I conclude that she's a vegetarian.

\begin{enumerate}
\item Make up a new piece of information about Sally that weakens the argument.
\item Make up a new piece of information about Sally that strengthens the argument.
\end{enumerate}


