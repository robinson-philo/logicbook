%from mod3-2, starting page 25

\subsection{Arguments with missing premises}
Quite often, an argument will not explicitly state a premise that we can see is
needed in order for the argument to be valid. In such a case, we can supply the
premise(s) needed in order so make the argument valid. Making missing
premises explicit is a central part of reconstructing arguments in standard form.
We have already dealt in part with this in the section on paraphrasing, but now
that we have introduced the concept of validity, we have a useful tool for
knowing when to supply missing premises in our reconstruction of an argument.
In some cases, the missing premise will be fairly obvious, as in the following:

\begin{quote}
Gary is a convicted sex-offender, so Gary is not allowed to work with
children.
\end{quote}

The premise and conclusion of this argument are straightforward:

\begin{quote}
\underline{Gary is a convicted sex-offender} \\
Therefore, Gary is not allowed to work with children (from premise 1) \\
\end{quote}

However, as stated, the argument is invalid. (Before reading on, see if you can
provide a counterexample for this argument. That is, come up with an imaginary
scenario in which the premise is true and yet the conclusion is false.) Here is just
one counterexample (there could be many): Gary is a convicted sex-offender but
the country in which he lives does not restrict convicted sex-offenders from
working with children. I don't know whether there are any such countries,
although I suspect there are (and it doesn't matter for the purpose of validity
whether there are or aren't). In any case, it seems clear that this argument is
relying upon a premise that isn't explicitly stated. We can and should state that
premise explicitly in our reconstruction of the standard form argument. But
what is the argument's missing premise? The obvious one is that no sexoffenders are allowed to work with children, but we could also use a more
carefully statement like this one:

\begin{quote}
Where Gary lives, no convicted sex-offenders are allowed to work with
children.
\end{quote}


It should be obvious why this is a more ``careful'' statement. It is more careful
because it is not so universal in scope, which means that it is easier for the
statement to be made true. By relativizing the statement that sex-offenders are
not allowed to work with children to the place where Gary lives, we leave open
the possibility that other places in the world don't have this same restriction. So
even if there are other places in the world where convicted sex-offenders are
allowed to work with children, our statements could still be true since in this
place (the place where Gary lives) they aren't. (For more on strong and weak
statements, see section 1.10). So here is the argument in standard form:

\begin{quote}
Gary is a convicted sex-offender. \\
\underline{Where Gary lives, no convicted sex-offenders are allowed to work with
children.} \\
Therefore, Gary is not allowed to work with children. (from premises 1-2) \\
\end{quote}

This argument is now valid: there is no way for the conclusion to be false,
assuming the truth of the premises. This was a fairly simple example where the
missing premise needed to make the argument valid was relatively easy to see.
As we can see from this example, a missing premise is a premise that the
argument needs in order to be as strong as possible. Typically, this means
supplying the statement(s) that are needed to make the argument valid. But in
addition to making the argument valid, we want to make the argument
plausible. This is called ``the principle of charity.'' The principle of charity
states that when reconstructing an argument, you should try to make that
argument (whether inductive or deductive) as strong as possible.
When it
comes to supplying missing premises, this means supplying the most plausible
premises needed in order to make the argument either valid (for deductive
arguments) or inductively strong (for inductive arguments).

Although in the last example figuring out the missing premise was relatively easy
to do, it is not always so easy. Here is an argument whose missing premises are
not as easy to determine:

\begin{quote}
Since children who are raised by gay couples often have psychological
and emotional problems, the state should discourage gay couples from
raising children.
\end{quote}

The conclusion of this argument, that the state should not allow gay marriage, is
apparently supported by a single premise, which should be recognizable from
the occurrence of the premise indicator, ``since.'' Thus, our initial reconstruction
of the standard form argument looks like this:

\begin{quote}
\underline{Children who are raised by gay couples often have psychological and
emotional problems}. \\
Therefore, the state should discourage gay couples from raising
children. \\
\end{quote}

However, as it stands, this argument is invalid because it depends on certain
missing premises. The conclusion of this argument is a normative statement—
a statement about whether something ought to be true, relative to some
standard of evaluation.

Normative statements can be contrasted with
descriptive statements, which are simply factual claims about what is true. For
example, ``Russia does not allow gay couples to raise children'' is a descriptive
statement. That is, it is simply a claim about what is in fact the case in Russia
today. In contrast, ``Russia should not allow gay couples to raise children'' is a
normative statement since it is not a claim about what is true, but what ought to
be true, relative to some standard of evaluation (for example, a moral or legal
standard). An important idea within philosophy, which is often traced back to
the Scottish philosopher David Hume (1711-1776), is that statements about what
ought to be the case (i.e., normative statements) can never be derived from
statements about what is the case (i.e., descriptive statements). This is known
within philosophy as the is-ought gap. The problem with the above argument
is that it attempts to infer a normative statement from a purely descriptive
statement, violating the is-ought gap. We can see the problem by constructing
a counterexample. Suppose that in society x it is true that children raised by gay
couples have psychological problems. However, suppose that in that society
people do not accept that the state should do what it can to decrease harm to
children. In this case, the conclusion, that the state should discourage gay
couples from raising children, does not follow. Thus, we can see that the
argument depends on a missing or assumed premise that is not explicitly stated.
That missing premise must be a normative statement, in order that we can infer
the conclusion, which is also a normative statement. There is an important
general lesson here: Many times an argument with a normative conclusion will
depend on a normative premise which is not explicitly stated. The missing
normative premise of this particular argument seems to be something like this:

\begin{quote}
The state should always do what it can to decrease harm to children.
\end{quote}


Notice that this is a normative statement, which is indicated by the use of the
word ``should.'' There are many other words that can be used to capture
normative statements such as: good, bad, and ought. Thus, we can reconstruct
the argument, filling in the missing normative premise like this:

\begin{quote}
Children who are raised by gay couples often have psychological and
emotional problems. \\
\underline{The state should always do what it can to decrease harm to children.} \\
Therefore, the state should discourage gay couples from raising
children. (from premises 1-2) \\
\end{quote}


However, although the argument is now in better shape, it is still invalid because
it is still possible for the premises to be true and yet the conclusion false. In
order to show this, we just have to imagine a scenario in which both the
premises are true and yet the conclusion is false. Here is one counterexample to
the argument (there are many). Suppose that while it is true that children of gay
couples often have psychological and emotional problems, the rate of
psychological problems in children raised by gay couples is actually lower than
in children raised by heterosexual couples. In this case, even if it were true that
the state should always do what it can to decrease harm to children, it does not
follow that the state should discourage gay couples from raising children. In
fact, in the scenario I've described, just the opposite would seem to follow: the
state should discourage heterosexual couples from raising children.

But even if we suppose that the rate of psychological problems in children of
gay couples is higher than in children of heterosexual couples, the conclusion
still doesn't seem to follow. For example, it could be that the reason that
children of gay couples have higher rates of psychological problems is that in a
society that is not yet accepting of gay couples, children of gay couples will face
more teasing, bullying and general lack of acceptance than children of
heterosexual couples. If this were true, then the harm to these children isn't so
much due to the fact that their parents are gay as it is to the fact that their
community does not accept them. In that case, the state should not necessarily
discourage gay couples from raising children. Here is an analogy: At one point
in our country's history (if not still today) it is plausible that the children of black
Americans suffered more psychologically and emotionally than the children of
white Americans. But for the government to discourage black Americans from
raising children would have been unjust, since it is likely that if there was a
higher incidence of psychological and emotional problems in black Americans,
then it was due to unjust and unequal conditions, not to the black parents, per
se. So, to return to our example, the state should only discourage gay couples
from raising children if they know that the higher incidence of psychological
problems in children of gay couples isn't the result of any kind of injustice, but is
due to the simple fact that the parents are gay.

Thus, one way of making the argument (at least closer to) valid would be to add
the following two missing premises:

\begin{quote}
A. The rate of psychological problems in children of gay couples is
higher than in children of heterosexual couples. \\
B. The higher incidence of psychological problems in children of gay
couples is not due to any kind of injustice in society, but to the fact
that the parents are gay.
\end{quote}

So the reconstructed standard form argument would look like this:

\begin{quote}
Children who are raised by gay couples often have psychological and
emotional problems. \\
The rate of psychological problems in children of gay couples is
higher than in children of heterosexual couples. \\
The higher incidence of psychological problems in children of gay
couples is not due to any kind of injustice in society, but to the fact
that the parents are gay. \\
\underline{The state should always do what it can to decrease harm to children}. \\
Therefore, the state should discourage gay couples from raising
children. (from premises 1-4) \\
\end{quote}

In this argument, premises 2-4 are the missing or assumed premises. Their
addition makes the argument much stronger, but making them explicit enables
us to clearly see what assumptions the argument relies on in order for the
argument to be valid. This is useful since we can now clearly see which premises
of the argument we may challenge as false. Arguably, premise 4 is false, since
the state shouldn't always do what it can to decrease harm to children. Rather,
it should only do so as long as such an action didn't violate other rights that the
state has to protect or create larger harms elsewhere.

The important lesson from this example is that supplying the missing premises
of an argument is not always a simple matter. In the example above, I have
used the principle of charity to supply missing premises. Mastering this skill is
truly an art (rather than a science) since there is never just one correct way of
doing it (cf. section 1.5) and because it requires a lot of skilled practice. \\



EXERCISES: \\
 
Supply the missing premise or premises needed in order to
make the following arguments valid. Try to make the premises as
plausible as possible while making the argument valid (which is to apply
the principle of charity).

1. Ed rides horses. Therefore, Ed is a cowboy.

2. Tom was driving over the speed limit. Therefore, Tom was doing
something wrong.

3. If it is raining then the ground is wet. Therefore, the ground must be
wet.

4. All elves drink Guinness, which is why Olaf drinks Guinness.

5. Mark didn't invite me to homecoming. Instead, he invited his friend
Alexia. So he must like Alexia more than me.

6. The watch must be broken because every time I have looked at it, the
hands have been in the same place.

7. Olaf drank too much Guinness and fell out of his second story
apartment window. Therefore, drinking too much Guinness caused
Olaf to injure himself.

8. Mark jumped into the air. Therefore, Mark landed back on the
ground.

9. In 2009 in the United States, the net worth of the median white
household was \$113,149 a year, whereas the net worth of the median
black household was \$5,677. Therefore, as of 2009, the United States
was still a racist nation.

10. The temperature of the water is 212 degrees Fahrenheit. Therefore,
the water is boiling.

11. Capital punishment sometimes takes innocent lives, such as the lives
of individuals who were later found to be not guilty. Therefore, we
should not allow capital punishment.

12. Allowing immigrants to migrate to the U.S. will take working class jobs
away from working class folks. Therefore, we should not allow
immigrants to migrate to the U.S.

13. Prostitution is a fair economic exchange between two consenting
adults. Therefore, prostitution should be allowed.

14. Colleges are more interested in making money off of their football
athletes than in educating them. Therefore, college football ought to
be banned.

15. Edward received an F in college Algebra. Therefore, Edward should
have studied more.

