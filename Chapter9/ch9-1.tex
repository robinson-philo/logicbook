
\chapter{Fallacies of Weak Induction}
As their name suggests, what these fallacies have in common is that they are bad -- that is, weak --
inductive arguments. Recall, inductive arguments attempt to provide premises that make their
conclusions more probable. We evaluate them according to how probable their conclusions are in
light of their premises: the more probable the conclusion (given the premises), the stronger the
argument; the less probable, the weaker. The fallacies of weak induction are arguments whose
premises do not make their conclusions very probable -- but that are nevertheless often successful
in convincing people of their conclusions. We will discuss five informal fallacies that fall under
this heading.

\subsubsection{Argument from Ignorance (Argumentum ad Ignorantiam)}
This is a particularly egregious and perverse fallacy. In essence, it's an inference from premises to
the effect that there's a lack of knowledge about some topic to a definite conclusion about that
topic. We don't know; therefore, we know!

Of course, put that baldly, it's plainly absurd; actual instances are more subtle. The fallacy comes
in a variety of closely related forms. It will be helpful to state them in bald/absurd schematic
fashion first, then elucidate with more subtle real-life examples.
The first form can be put like this:

\begin{quote}
\underline{Nobody knows how to explain phenomenon X.} \\
My crazy theory about X is true.\end{quote}

That sounds silly, but consider an example: those ``documentary'' programs on cable TV about
aliens. You know, the ones where they suggest that extraterrestrials built the pyramids or
something (there are books and websites, too). How do they get you to believe that crazy theory?
By creating mystery! By pointing to facts that nobody can explain. The Great Pyramid at Giza is
aligned (almost) exactly with the magnetic north pole! On the day of the summer solstice, the sun
sets exactly between two of the pyramids! The height of the Great Pyramid is (almost) exactly one
one-millionth the distance from the Earth to the Sun! How could the ancient Egyptians have such
sophisticated astronomical and geometrical knowledge? Why did the Egyptians, careful 
recordkeepers in (most) other respects, (apparently) not keep detailed records of the construction of the
pyramids? Nobody knows. Conclusion: aliens built the pyramids.
In other words, there are all sorts of (sort of) surprising facts about the pyramids, and nobody
knows how to explain them. From these premises, which establish only our ignorance, we're
encouraged to conclude that we know something: aliens built the pyramids. That's quite a leap --
too much of a leap.

Another form this fallacy takes can be put crudely thus:

Nobody can PROVE that I'm wrong.
I'm right.

The word `prove' is in all-caps because stressing it is the key to this fallacious argument: the
standard of proof is set impossibly high, so that almost no amount of evidence would constitute a
refutation of the conclusion.

An example will help. There are lots of people who claim that evolutionary biology is a lie: there's
no such thing as evolution by natural selection, and it's especially false to claim that humans
evolved from earlier species, that we share a common ancestor with apes. Rather, the story goes,
the Bible is literally true: the Earth is only about 6,000 years old, and humans were created as-is
by God just as the Book of Genesis describes. The Argument from Ignorance is one of the favored
techniques of proponents of this view. They are especially fond of pointing to `gaps' in the fossil
record -- the so-called `missing link' between humans and a pre-human, ape-like species and
claim that the incompleteness of the fossil record vindicates their position.

But this argument is an instance of the fallacy. The standard of proof -- a complete fossil record
without any gaps -- is impossibly high. Evolution has been going on for a LONG time (the Earth
is actually about 4.5 billion years old, and living things have been around for at least 3.5 billion
years). So many species have appeared and disappeared over time that it's absurd to think that we
could even come close to collecting fossilized remains of anything but the tiniest fraction of them.
It's hard to become a fossil, after all: a creature has to die under special circumstances to even have
a chance for its remains to do anything than turn into compost. And we haven't been searching for
fossils in a systematic way for very long (only since the mid-1800s or so). It's no surprise that
there are gaps in the fossil record, then. What's surprising, in fact, is that we have as rich a fossil
record as we do. Many, many transitional species have been discovered, both between humans and
their ape-like ancestors, and between other modern species and their distant forbears (whales used
to be land-based creatures, for example; we know this (in part) from the fossils of early protowhale species with longer and longer rear hip- and leg-bones).
We will never have a fossil record complete enough to satisfy skeptics of evolution. But their
standard is unreasonably high, so their argument is fallacious. Sometimes they put it even more
simply: nobody was around to witness evolution in action; therefore, it didn't happen. This is
patently absurd, but it follows the same pattern: an unreasonable standard of proof (witnesses to
evolution in action; impossible, since it takes place over such a long period of time), followed by
the leap to the unwarranted conclusion.

Yet another version of the Argument from Ignorance goes like this:

\begin{quote}
\underline{I can't imagine/understand how X could be true.} \\
X is false.
\end{quote}

Of course lack of imagination on the part of an individual isn't evidence for or against a
proposition, but people often argue this way. A (hilarious) example comes from the rap duo Insane
Clown Posse in their 2009 single, `Miracles'. Here's the line:

\begin{quote}Water, fire, air and dirt \\
F**king magnets, how do they work? \\
And I don't wanna talk to a scientist \\
Ya'll mother**kers lying, and getting me pissed.
\end{quote}

Violent J and Shaggy 2 Dope can't understand how there could be a scientific, non-miraculous
explanation for the workings of magnets. They conclude, therefore, that magnets are miraculous.
A final form of the Argument from Ignorance can be put crudely thus:

\begin{quote}\underline{No evidence has been found that X is true.} \\
X is false.\end{quote}

You may have heard the slogan, ``Absence of evidence is not evidence of absence.'' This is an
attempt to sum up this version of the fallacy. But it's not quite right. What it should say is that
absence of evidence is not always definitive evidence of absence. An example will help illustrate
the idea. During the 2016 presidential campaign, a reporter (David Fahrentold) took to Twitter to
announce that despite having ``spent weeks looking for proof that [Donald Trump] really does give
millions of his own [money] to charity \dots '' he could only find one donation, to the NYC Police
Athletic League. Trump has claimed to have given millions of dollars to charities over the years.
Does this reporter's failure to find evidence of such giving prove that Trump's claims about his
charitable donations are false? No. To rely only on this reporter's testimony to draw such a
conclusion would be to commit the fallacy.

However, the failure to uncover evidence of charitable giving does provide some reason to suspect
Trump's claims may be false. How much of a reason depends on the reporter's methods and
credibility, among other things.\footnote{And, in fact, Fahrentold subsequently performed and documented (in the Washington Post on 9/12/16) a rather
exhaustive unsuccessful search for evidence of charitable giving, providing strong support for the conclusion that
Trump didn't give as he'd claimed.}
But sometimes a lack of evidence can provide strong support for
a negative conclusion. This is an inductive argument; it can be weak or strong. For example,
despite multiple claims over many years (centuries, if some sources can be believed), no evidence
has been found that there's a sea monster living in Loch Ness in Scotland. Given the size of the
body of water, and the extensiveness of the searches, this is pretty good evidence that there's no
such creature -- a strong inductive argument to that conclusion. To claim otherwise -- that there is
such a monster, despite the lack of evidence -- would be to commit the version of the fallacy
whereby one argues ``You can't PROVE I'm wrong; therefore, I'm right,'' where the standard of
proof is unreasonably high.

One final note on this fallacy: it's common for people to mislabel certain bad arguments as
arguments from ignorance; namely, arguments made by people who obviously don't know what
the heck they're talking about. People who are confused or ignorant about the subject on which
they're offering an opinion are liable to make bad arguments, but the fact of their ignorance is not
enough to label those arguments as instances of the fallacy. We reserve that designation for
arguments that take the forms canvassed above: those that rely on ignorance -- and not just that of
the arguer, but of the audience as well -- as a premise to support the conclusion.

\subsubsection{Appeal to Inappropriate Authority}
One way of making an inductive argument -- of lending more credence to your conclusion -- is to
point to the fact that some relevant authority figure agrees with you. In law, for example, this kind
of argument is indispensable: appeal to precedent (Supreme Court rulings, etc.) is the attorney's
bread and butter. And in other contexts, this kind of move can make for a strong inductive
argument. If I'm trying to convince you that fluoridated drinking water is safe and beneficial, I can
point to the Centers for Disease Control, where a wealth of information supporting that claim can
be found.\footnote{Check it out: https://www.cdc.gov/fluoridation/} 
Those people are scientists and doctors who study this stuff for a living; they know
what they’re talking about.

One commits the fallacy when one points to the testimony of someone who's not an authority on
the issue at hand. This is a favorite technique of advertisers. We've all seen celebrity endorsements
of various products. Sometimes the celebrities are appropriate authorities: there was a Buick
commercial from 2012 featuring Shaquille O'Neal, the Hall of Fame basketball player, testifying
to the roominess of the car's interior (despite its compact size). Shaq, a very, very large man, is an
appropriate authority on the roominess of cars! But when Tiger Woods was shilling for Buicks a
few years earlier, it wasn't at all clear that he had any expertise to offer about their merits relative
to other cars. Woods was an inappropriate authority; those ads committed the fallacy.
Usually, the inappropriateness of the authority being appealed to is obvious. But sometimes it isn't.
A particularly subtle example is AstraZeneca's hiring of Dr. Phil McGraw in 2016 as a
spokesperson for their diabetes outreach campaign. AstraZeneca is a drug manufacturing
company. They make a diabetes drug called Bydureon. The aim of the outreach campaign,
ostensibly, is to increase awareness among the public about diabetes; but of course the real aim is
to sell more Bydureon. A celebrity like Dr. Phil can help. Is he an appropriate authority? That's a
hard question to answer. It's true that Dr. Phil had suffered from diabetes himself for 25 years, and
that he personally takes the medication. So that's a mark in his favor, authority-wise. But is that
enough? We'll talk about how feeble Phil's sort of anecdotal evidence is in supporting general
claims (in this case, about a drug's effectiveness) when we discuss the hasty generalization fallacy;
suffice it to say, one person's positive experience doesn't prove that the drug is effective. But, Dr.
Phil isn't just a person who suffers from diabetes; he's a doctor! It's right there in his name
(everybody always simply refers to him as ``Dr. Phil''). Surely that makes him an appropriate
authority on the question of drug effectiveness. Or maybe not. Phil McGraw is not a medical
doctor; he's a PhD. He has a doctorate in Psychology. He's not a licensed psychologist; he cannot
legally prescribe medication. He has no relevant professional expertise about drugs and their
effectiveness. He is not an appropriate authority in this case. He looks like one, though, which
makes this a very sneaky, but effective, advertising campaign.

\subsubsection{Post hoc ergo propter hoc}
Here's another fallacy for which people always use the Latin, usually shortening it to ``post hoc''.
The whole phrase translates to ``After this, therefore because of this'', which is a pretty good
summation of the pattern of reasoning involved. Crudely and schematically, it looks like this:

\begin{quote}\underline{X occurred before Y.} \\
X caused Y.\end{quote}

This is not a good inductive argument. That one event occurred before another gives you some
reason to believe it might be the cause -- after all, X can't cause Y if it happened after Y did -- but
not nearly enough to conclude that it is the cause. A silly example: I, your humble author, was
born on June 19th, 1974; this was just shortly before a momentous historical event, Richard Nixon's
resignation of the Presidency on August 9th later that summer. My birth occurred before Nixon's
resignation; but this is (obviously!) not a reason to think that it caused his resignation.
Though this kind of reasoning is obviously shoddy -- a mere temporal relationship clearly does not
imply a causal relationship -- it is used surprisingly often. In 2012, New York Yankees shortstop
Derek Jeter broke his ankle. It just so happened that this event occurred immediately after another
event, as Donald Trump pointed out on Twitter: ``Derek Jeter broke ankle one day after he sold his
apartment in Trump World Tower.'' Trump followed up: ``Derek Jeter had a great career until 3
days ago when he sold his apartment at Trump World Tower - I told him not to sell - karma?'' No, not karma; just bad luck.

Nowhere is this fallacy more in evidence than in our evaluation of the performance of presidents
of the United States. Everything that happens during or immediately after their administrations
tends to be pinned on them. But presidents aren't all-powerful; they don't cause everything that
happens during their presidencies. On July 9th, 2016, a short piece appeared in the Washington
Post with the headline ``Police are safer under Obama than they have been in decades''. What does
a president have to do with the safety of cops? Very little, especially compared to other factors like
poverty, crime rates, policing practices, rates of gun ownership, etc., etc., etc. To be fair, the article
was aiming to counter the equally fallacious claims that increased violence against police was
somehow caused by Obama. Another example: in October 2015, US News \& World Report
published an article asking (and purporting to answer) the question, ``Which Presidents Have Been
Best for the Economy?'' It had charts listing GDP growth during each administration since
Eisenhower. But while presidents and their policies might have some effect on economic growth,
their influence is certainly swamped by other factors. Similar claims on behalf of state governors
are even more absurd. At the 2016 Republican National Convention, Governors Scott Walker and
Mike Pence -- of Wisconsin and Indiana, respectively -- both pointed to record-high employment
in their states as vindication of their conservative, Republican policies. But some other states were
also experiencing record-high employment at the time: California, Minnesota, New Hampshire,
New York, Washington. Yes, they were all controlled by Democrats. Maybe there's a separate
cause for those strong jobs numbers in differently governed states? Possibly it has something to
do with the improving economy and overall health of the job market in the whole country?

\subsubsection{Hasty Generalization}
Many inductive arguments involve an inference from particular premises to a general conclusion;
this is generalization. For example, if you make a bunch of observations every morning that the
sun rises in the east, and conclude on that basis that, in general, the sun always rises in the east,
this is a generalization. And it's a good one! With all those particular sunrise observations as
premises, your conclusion that the sun always rises in the east has a lot of support; that's a strong
inductive argument.

One commits the hasty generalization fallacy when one makes this kind of inference based on an
insufficient number of particular premises, when one is too quick -- hasty -- in inferring the general
conclusion.

People who deny that global warming is a genuine phenomenon often commit this fallacy. In
February of 2015, the weather was unusually cold in Washington, DC. Senator James Inhofe of
Oklahoma famously took to the Senate floor wielding a snowball. ``In case we have forgotten,
because we keep hearing that 2014 has been the warmest year on record, I ask the chair, `You
know what this is?' It's a snowball, from outside here. So it's very, very cold out. Very
unseasonable.'' He then tossed the snowball at his colleague, Senator Bill Cassidy of Louisiana,
who was presiding over the debate, saying, ``Catch this.''
Senator Inhofe commits the hasty generalization fallacy. He's trying to establish a general
conclusion -- that 2014 wasn't the warmest year on record, or that global warming isn't really
happening (he's on the record that he considers it a ``hoax''). But the evidence he presents is
insufficient to support such a claim. His evidence is an unseasonable coldness in a single place on
the planet, on a single day. We can't derive from that any conclusions about what's happening,
temperature-wise, on the entire planet, over a long period of time. That the earth is warming is not
a claim that everywhere, at every time, it will always be warmer than it was; the claim is that, on
average, across the globe, temperatures are rising. This is compatible with a couple of cold snaps
in the nation's capital.

Many people are susceptible to hasty generalizations in their everyday lives. When we rely on
anecdotal evidence to make decisions, we commit the fallacy. Suppose you're thinking of buying
a new car, and you're considering a Subaru. Your neighbor has a Subaru. So what do you do? You
ask your neighbor how he likes his Subaru. He tells you it runs great, hasn't given him any trouble.
You then, fallaciously, conclude that Subarus must be terrific cars. But one person's testimony
isn't enough to justify that conclusion; you'd need to look at many, many more drivers'
experiences to reach such a conclusion (this is why the magazine Consumer Reports is so useful).
A particularly pernicious instantiation of the Hasty Generalization fallacy is the development of
negative stereotypes. People often make general claims about religious or racial groups, ethnicities
and nationalities, based on very little experience with them. If you once got mugged by a Puerto
Rican, that's not a good reason to think that, in general, Puerto Ricans are crooks. If a waiter at a
restaurant in Paris was snooty, that's no reason to think that French people are stuck up. And yet
we see this sort of faulty reasoning all the time.

\subsubsection{Slippery Slope}
Like the post hoc fallacy, the slippery slope fallacy is a weak inductive argument to a conclusion
about causation. This fallacy involves making an insufficiently supported claim that a certain
action or event will set off an unstoppable causal chain-reaction -- putting us on a slippery slope --
leading to some disastrous effect.

This style of argument was a favorite tactic of religious conservatives who opposed gay marriage.
They claimed that legalizing same-sex marriage would put the nation on a slippery slope to
disaster. Famous Christian leader Pat Robertson, on his television program The 700 Club, puts the
case nicely. When asked about gay marriage, he responded with this:

We haven't taken this to its ultimate conclusion. You've got polygamy out there. How can
we rule that polygamy is illegal when you say that homosexual marriage is legal? What is
it about polygamy that's different? Well, polygamy was outlawed because it was
considered immortal according to Biblical standards. But if we take Biblical standards
away in homosexuality, well what about the other? And what about bestiality? And
ultimately what about child molestation and pedophilia? How can we criminalize these
things, at the same time have Constitutional amendments allowing same-sex marriage
among homosexuals? You mark my words, this is just the beginning of a long downward
slide in relation to all the things that we consider to be abhorrent.

This a classic slippery slope fallacy; he even uses the phrase ``long downward slide''! The claim is
that allowing gay marriage will force us to decriminalize polygamy, bestiality, child molestation,
pedophilia -- and ultimately, ``all the things that we consider to be abhorrent.'' Yikes! That's a lot
of things. Apparently, gay marriage will lead to utter anarchy.
There are genuine slippery slopes out there -- unstoppable causal chain-reactions. But this isn't one
of them. The mark of the slippery slope fallacy is the assertion that the chain can't be stopped,
with reasons that are insufficient to back up that assertion. In this case, Pat Robertson has given us
the abandonment of ``Biblical standards'' as the lubrication for the slippery slope. But this is
obviously insufficient. Biblical standards are expressly forbidden, by the ``establishment clause''
of the First Amendment to the U.S. Constitution, from forming the basis of the legal code. The
slope is not slippery. As recent history has shown, the legalization of same sex marriage does not
lead to the acceptance of bestiality and pedophilia; the argument is fallacious.
Fallacious slippery slope arguments have long been deployed to resist social change. Those
opposed to the abolition of slavery warned of economic collapse and social chaos. Those who
opposed women's suffrage asserted that it would lead to the dissolution of the family, rampant
sexual promiscuity, and social anarchy. Of course none of these dire predictions came true; the
slopes simply weren't slippery.

