%pg63 mod4-3 ch3.pdf
\chapter{Induction}

\subsection{Inductive argumentation}
Inductive argumentation is a less certain, more realistic, more familiar way of
reasoning that we all do, all the time. Inductive argumentation recognizes, for instance, that a premise like
``All horses have four legs'' comes from our previous
experience of horses. If one day we were to encounter
a three-legged horse, deductive logic would tell us that
``All horses have four legs'' is false, at which point the
premise becomes rather useless for a deducer. In fact,
deductive logic tells us that if the premise ``All horses
have four legs'' is false, even if we know there are many,
many four-legged horses in the world, when we go
to the track and see hordes of four-legged horses, all
we can really be certain of is that ``There is at least one
four-legged horse.''

     Inductive logic allows for the more realistic
premise, ``The vast majority of horses have four legs''.
And inductive logic can use this premise to infer other
useful information, like ``If I'm going to get Chestnut
booties for Christmas, I should probably get four of
them.'' The trick is to recognize a certain amount of
uncertainty in the truth of the conclusion, something
for which deductive logic does not allow. In real life,
however, inductive logic is used much more frequently
and (hopefully) with some success. 

\subsubsection{Predicting the Future}

We constantly use inductive reasoning to predict the
future. We do this by compiling evidence based on
past observations, and by assuming that the future will
resemble the past. For instance, I make the observation
that every other time I have gone to sleep at night,
I have woken up in the morning. There is actually
no certainty that this will happen, but I make the
inference because of the fact that this is what has happened every other time. In fact, it is not the case that
``All people who go to sleep at night wake up in the
morning''. But I'm not going to lose any sleep over that.
And we do the same thing when our experience has
been less consistent. For instance, I might make the assumption that, if there's someone at the door, the dog
will bark. But it's not outside the realm of possibility
that the dog is asleep, has gone out for a walk, or has
been persuaded not to bark by a clever intruder with
sedative-laced bacon. I make the assumption that if
there's someone at the door, the dog will bark, because
that is what usually happens.

\subsubsection{Explaining Common Occurrences}

We also use inductive reasoning to explain things
that commonly happen. For instance, if I'm about to
start an exam and notice that Bill is not here, I might
explain this to myself with the reason that Bill is stuck
in traffic. I might base this on the reasoning that being
stuck in traffic is a common excuse for being late, or
because I know that Bill never accounts for traffic
when he's estimating how long it will take him to get
somewhere. Again, that Bill is actually stuck in traffic
is not certain, but I have some good reasons to think
it's probable. We use this kind of reasoning to explain
past events as well. For instance, if I read somewhere
that 1986 was a particularly good year for tomatoes,
I assume that 1986 also had some ideal combination
of rainfall, sun, and consistently warm temperatures.
Although it's possible that a scientific madman circled
the globe planting tomatoes wherever he could in
1986, inductive reasoning would tell me that the
former, environmental explanation is more likely. (But
I could be wrong.)


\subsubsection{Generalizing}

Often we would like to make general claims, but in
fact it would be very difficult to prove any general
claim with any certainty. The only way to do so would
be to observe every single case of something about
which we wanted to make an observation. This would
be, in fact, the only way to prove such assertions as,
``All swans are white''. Without being able to observe
every single swan in the universe, I can never make
that claim with certainty. Inductive logic, on the other
hand, allows us to make the claim, with a certain
amount of modesty.

\subsection{Inductive Generalization}

Inductive generalization allows us to make general
claims, despite being unable to actually observe every
single member of a class in order to make a certainly
true general statement. We see this in scientific studies,
population surveys, and in our own everyday reasoning. Take for example a drug study. Some doctor or
other wants to know how many people will go blind
if they take a certain amount of some drug for so
many years. If they determine that 5\% 
of people in the
study go blind, they then assume that 5\% 
of all people
who take the drug for that many years will go blind.
Likewise, if I survey a random group of people and ask
them what their favourite color is, and 75\% 
of them
say ``purple'', then I assume that purple is the favourite
colour of 75\% 
of people. But we have to be careful
when we make an inductive generalization. When you
tell me that 75\% 
of people really like purple, I'm going
to want to know whether you took that survey outside
a Justin Bieber concert.


     Let's take an example. Let's say I asked a class of
400 students whether or not they think logic is a valuable course, and 90\% 
of them said yes. I can make an
inductive argument like this: \\

\begin{quote}
\underline{90\% of 400 students believe that logic is a valuable
course}. \\
Therefore 90\% of all students believe that logic is a
valuable course. \\
\end{quote}

There are certain things I need to take into
account in judging the quality of this argument.
For instance, did I ask this in a logic course? Did the
respondents have to raise their hands so that the
professor could see them, or was the survey taken
anonymously? Are there enough students in the course
to justify using them as a representative group for
students in general?

If I did, in fact, make a class of 400 logic students
raise their hands in response to the question of
whether logic is valuable course, then we can identify
a couple of problems with this argument. The first is
bias. We can assume that anyone enrolled in a logic
course is more likely to see it as valuable than any
random student. I have therefore skewed the argument
in favour of logic courses. I can also question whether
the students were answering the question honestly. Perhaps if they are trying to save the professor's feelings,
they are more likely to raise their hands and assure her
that the logic course is a valuable one.


Now let's say I've avoided those problems. I have
assured that the 400 students I have asked are randomly selected, say, by soliciting email responses from
randomly selected students from the university's entire
student population. Then the argument looks stronger.

Another problem we might have with the
argument is whether I have asked enough students so
that the whole population is well-represented. If the
student body as a whole consists of 400 students, my
argument is very strong. If the student body numbers
in the tens of thousands, I might want to ask a few
more before assuming that the opinions of a few mirror those of the many. This would be a problem with
my sample size.

Let's take another example. Now I'm going to run
a scientific study, in which I will pay someone \$50 to
take a drug with unknown effects and see if it makes
them blind. In order to control for other variables, I
open the study only to white males between the ages
of 18 and 25.

A bad inductive argument would say:

\begin{quote}
\underline{40\% of 1000 people who took the drug went blind}. \\
40\% of people who take the drug will go blind. \\
\end{quote}

A better inductive argument would make a more
modest claim:

\begin{quote}
\underline{40\% of the 1000 people who took the drug went blind}. \\
40\% of white males between the ages of 18 and 25 who take the drug will go blind. \\
\end{quote}
    
The point behind this example is to show how inductive reasoning imposes an important limitation on
the possible conclusions a study or a survey can make.
In order to make good generalizations, we need to
ensure that our sample is representative, non-biased,
and sufficiently sized.


\subsection{Statistical Syllogism}

Where in an inductive generalization we saw statement expressing a statistic applied to a more general
group, we can also use statistics to go from the general
to the particular. For instance, if I know that most computer science majors are male, and that some random
individual with the androgynous name ``Cameron'' is
an computer science major, then we can be reasonably
certain that Cameron is a male. We tend to represent
the uncertainty by qualifying the conclusion with the
word ``probably''. If, on the other hand, we wanted to
say that something is unlikely, like that Cameron
were a female, we could use ``probably not''. It is also
possible to temper our conclusion with other similar
qualifying words.


Let's take an example.

\begin{quote}
Of the 133 people found guilty of homicide last year in Canada, 79\% were jailed. \\
\underline{Socrates was found guilty of homicide last year in Canada}. \\
Therefore, Socrates was probably jailed.
\end{quote}


In this case we can be reasonably sure that
Socrates is currently rotting in prison. Now the
certainty of our conclusion seems to be dependent on
the statistics we're dealing with. There are definitely
more certain and more uncertain cases.

\begin{quote}
In the last election, 50\% of voting Americans voted for Obama, while 48\% voted for Romney. \\
\underline{Jim is a voting American}. \\
Jim probably voted for Obama. \\
\end{quote}


Clearly, this argument is not as strong as the first.
It is only slightly more likely than not that Jim voted
for Obama. In this case we might want to revise our
conclusion to say:

\begin{quote}
(C) It is slightly more likely than not that Jim
voted for Obama. \\
\end{quote}

In other cases, the likelihood that something is or
is not the case approaches certainty. For example:

\begin{quote}
There is a 0.00000059\% chance you will die on any
single flight, assuming you use one of the most poorly
rated airlines. \\
\underline{I'm flying to Paris next week}. \\
There's more than a million to one chance that I will
die on my flight.\\
\end{quote}


Note that in all of these examples, nothing is ever
stated with absolute certainty. It is possible to improve
the chances that our conclusions will be accurate by
being more specific, or finding out more information.
We would know more about Jim's voting strategy,
for instance, if we knew where he lived, his previous
voting habits, or if we simply asked him for whom he
voted (in which case, we might also want to know how
often Jim lies).

