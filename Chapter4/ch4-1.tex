%from p139, ch3 of module4-1-ch3.pdf
\section{Potential Problems with Inductive arguments and statistical generalizations}
As we've seen, an inductive argument is an argument
whose conclusion is supposed to follow from its premises with a high level of
probability, rather than with certainty. This means that although it is possible
that the conclusion doesn't follow from its premises, it is unlikely that this is the
case. We said that inductive arguments are ``defeasible," meaning that we
could turn a strong inductive argument into a weak inductive argument simply
by adding further premises to the argument. In contrast, deductive arguments
that are valid can never be made invalid by adding further premises. Consider our
``Tweets" argument:\\

\begin{quote}
Tweets is a healthy, normally functioning bird\\
\underline{Most healthy, normally functioning birds fly} \\
Therefore, Tweets probably flies \\
\end{quote}

Without knowing anything else about Tweets, it is a good bet that Tweets flies.
However, if we were to add that Tweets is 6 ft. tall and can run 30 mph, then it is
no longer a good bet that Tweets can fly (since in this case Tweets is likely an
ostrich and therefore can't fly). The second premise, ``most healthy, normally
functioning birds fly," is a statistical generalization. Statistical generalizations
are generalizations arrived at by empirical observations of certain regularities.
Statistical generalizations can be either universal or partial.
Universal
generalizations assert that all members (i.e., 100\%) 
of a certain class have a
certain feature, whereas partial generalizations assert that most or some
percentage of members of a class have a certain feature. For example, the
claim that ``67.5\% 
of all prisoners released from prison are rearrested within
three years" is a partial generalization that is much more precise than simply
saying that ``most prisoners released from prison are rearrested within three
years." In contrast, the claim that ``all prisoners released from prison are
rearrested within three years" is a universal generalization. As we can see from
these examples, deductive arguments typically use universal statistical
generalizations whereas inductive arguments typically use partial statistical
generalizations. Since statistical generalizations are often crucial premises in
both deductive and inductive arguments, being able to evaluate when a
statistical generalization is good or bad is crucial for being able to evaluate
arguments. What we are doing in evaluating statistical generalizations is
determining whether the premise in our argument is true (or at least wellsupported 
by the evidence). For example, consider the following inductive
argument, whose premise is a (partial) statistical generalization:\\

\begin{quote}
\underline{70\% of voters say they will vote for candidate X} \\
Therefore, candidate X will probably win the election \\
\end{quote}

This is an inductive argument because even if the premise is true, the conclusion
could still be false (for example, an opponent of candidate X could
systematically kill or intimidate those voters who intend to vote for candidate X
so that very few of them will actually vote). Furthermore, it is clear that the
argument is intended to be inductive because the conclusion contains the word
``probably," which clearly indicates that an inductive, rather than deductive,
inference is intended. Remember that in evaluating arguments we want to know
about the strength of the inference from the premises to the conclusion, but we
also want to know whether the premise is true! We can assess whether or not a
statistical generalization is true by considering whether the statistical
generalization meets certain conditions. There are two conditions that any
statistical generalization must meet in order for the generalization to be deemed
``good." 

\begin{quote}
1. Adequate sample size: the sample size must be large enough to
support the generalization. \\
2. Non-biased sample: the sample must not be biased. \\
\end{quote}

A sample is simply a portion of a population. A population is the totality of
members of some specified set of objects or events. For example, if I were
determining the relative proportion of cars to trucks that drive down my street
on a given day, the population would be the total number of cars and trucks that
drive down my street on a given day. If I were to sit on my front porch from 
12-2 pm and count all the cars and trucks that drove down my street, that would be
a sample. A good statistical generalization is one in which the sample is
representative of the population. When a sample is representative, the
characteristics of the sample match the characteristics of the population at large.
For example, my method of sampling cars and trucks that drive down my street
would be a good method as long as the proportion of trucks to cars that drove
down my street between 12-2 pm matched the proportion of trucks to cars that
drove down my street during the whole day. If for some reason the number of
trucks that drove down my street from 12-2 pm was much higher than the
average for the whole day, my sample would not be representative of the
population I was trying to generalize about (i.e., the total number of cars and
trucks that drove down my street in a day). The ``adequate sample size"
condition and the ``non-biased sample" condition are ways of making sure that a
sample is representative. In the rest of this section, we will explain each of these
conditions in turn.

It is perhaps easiest to illustrate these two conditions by considering what is
wrong with statistical generalizations that fail to meet one or more of these
conditions. First, consider a case in which the sample size is too small (and thus
the adequate sample size condition is not met). If I were to sit in front of my
house for only fifteen minutes from 12:00-12:15 and saw only one car, then my
sample would consist of only 1 automobile, which happened to be a car. If I
were to try to generalize from that sample, then I would have to say that only
cars (and no trucks) drive down my street. But the evidence for this universal
statistical generalization (i.e., ``every automobile that drives down my street is a
car") is extremely poor since I have sampled only a very small portion of the
total population (i.e., the total number of automobiles that drive down my
street). Taking this sample to be representative would be like going to Flagstaff,
AZ for one day and saying that since it rained there on that day, it must rain
every day in Flagstaff. Inferring to such a generalization is an informal fallacy
called ``hasty generalization." One commits the fallacy of hasty generalization
when one infers a statistical generalization (either universal or partial) about a
population from too few instances of that population. Hasty generalization
fallacies are very common in everyday discourse, as when a person gives just
one example of a phenomenon occurring and implicitly treats that one case as
sufficient evidence for a generalization. This works especially well when fear or
practical interests are involved. For example, Jones and Smith are talking about
the relative quality of Fords versus Chevys and Jones tells Smith about his
uncle's Ford, which broke down numerous times within the first year of owning
it. Jones then says that Fords are just unreliable and that that is why he would
never buy one. The generalization, which is here ambiguous between a
universal generalization (i.e., all Fords are unreliable) and a partial generalization
(i.e., most/many Fords are unreliable), is not supported by just one case,
however convinced Smith might be after hearing the anecdote about Jones's
uncle's Ford.

The non-biased sample condition may not be met even when the adequate
sample size condition is met. For example, suppose that I count all the cars on
my street for a three hour period from 11-2 pm during a weekday. Let's assume
that counting for three hours straight give us an adequate sample size.
However, suppose that during those hours (lunch hours) there is a much higher
proportion of trucks to cars, since (let's suppose) many work trucks are coming
to and from worksites during those lunch hours. If that were the case, then my
sample, although large enough, would not be representative because it would
be biased. In particular, the number of trucks to cars in the sample would be
higher than in the overall population, which would make the sample
unrepresentative of the population (and hence biased).

Another good way of illustrating sampling bias is by considering polls. So
consider candidate X who is running for elected office and who strongly
supports gun rights and is the candidate of choice of the NRA. Suppose an
organization runs a poll to determine how candidate X is faring against
candidate Y, who is actively anti gun rights. But suppose that the way the
organization administers the poll is by polling subscribers to the magazine, Field
and Stream. Suppose the poll returned over 5000 responses, which, let's
suppose, is an adequate sample size and out of those responses, 89\% 
favored
candidate X. If the organization were to take that sample to support the
statistical generalization that ``most voters are in favor of candidate X" then they
would have made a mistake. If you know anything about the magazine Field
and Stream, it should be obvious why. Field and Stream is a magazine whose
subscribers who would tend to own guns and support gun rights. Thus we
would expect that subscribers to that magazine would have a much higher
percentage of gun rights activists than would the general population, to which
the poll is attempting to generalize. But in this case, the sample would be
unrepresentative and biased and thus the poll would be useless. Although the
sample would allow us to generalize to the population, ``Field and Stream
subscribers," it would not allow us to generalize to the population at large.
Let's consider one more example of a sampling bias. Suppose candidate X
were running in a district in which there was a high proportion of elderly voters.
Suppose that candidate X favored policies that elderly voters were against. For
example, suppose candidate X favors slashing Medicare funding to reduce the
budget deficit, whereas candidate Y favored maintaining or increasing support
to Medicare. Along comes an organization who is interested in polling voters to
determine which candidate is favored in the district. Suppose that the
organization chooses to administer the poll via text message and that the results
of the poll show that 75\% 
of the voters favor candidate X. Can you see what's
wrong with the poll--why it is biased? You probably recognize that this polling
method will not produce a representative sample because elderly voters are
much less likely to use cell phones and text messaging and so the poll will leave
out the responses of these elderly voters (who, we've assumed make up a large
segment of the population).

Thus, the sample will be biased and
unrepresentative of the target population. As a result, any attempt to generalize
to the general population would be extremely ill-advised. \\

EXERCISES \\

What kinds of problems, if any, do the following statistical
generalizations have? If there is a problem with the generalization,
specify which of the two conditions (adequate sample size, non-biased
sample) are not met. Some generalizations may have multiple problems.
If so, specify all of the problems you see with the generalization.

\begin{enumerate}
\item Bob, from Silverton, CO drives a 4x4 pickup truck, so most people
from Silverton, CO drive 4x4 pickup trucks.

\item Tom counts and categorizes birds that land in the tree in his backyard
every morning from 5:00-5:20 am. He counts mostly morning doves
and generalizes, ``most birds that land in my tree in the morning are
morning doves."

\item Tom counts and categorizes birds that land in the tree in his backyard
every morning from 5:00-6:00 am. He counts mostly morning doves
and generalizes, ``most birds that land in my tree during the 24-hour
day are morning doves."

\item Tom counts and categorizes birds that land in the tree in his backyard
every day from 5:00-6:00 am, from 11:00-12:00 pm, and from 5:006:00 pm. He counts mostly morning doves and generalizes, ``most
birds that land in my tree during the 24-hour day are morning doves."

\item Tom counts and categorizes birds that land in the tree in his backyard
every evening from 10:00-11:00 pm. He counts mostly owls and
generalizes, ``most birds that land in my tree throughout the 24-hour
day are owls."

\item Tom counts and categorizes birds that land in the tree in his backyard
every evening from 10:00-11:00 pm and from 2:00-3:00 am. He
counts mostly owls and generalizes, ``most birds that land in my tree
throughout the night are owls."

\item A poll administered to 10,000 registered voters who were homeowners showed that 90\% 
supported a policy to slash Medicaid
funding and decrease property taxes. Therefore, 90\% 
of voters
support a policy to slash Medicaid funding.

\item A telephone poll administered by a computer randomly generating
numbers to call, found that 68\% 
of Americans in the sample of 2000
were in favor of legalizing recreational marijuana use. Thus, almost
70\% 
of Americans favor legalizing recreation marijuana use.

\item A randomized telephone poll in the United States asked respondents
whether they supported a) a policy that allows killing innocent children
in the womb or b) a policy that saves the lives of innocent children in
the womb. The results showed that 69\% 
of respondents choose
option ``b" over option ``a." The generalization was made that ``most
Americans favor a policy that disallows abortion."

\item Steve's first rock and roll concert was an Ani Difranco concert, in which
most of the concert-goers were women with feminist political slogans
written on their t-shirts. Steve makes the generalization that ``most
rock and roll concert-goers are women who are feminists." He then
applies this generalization to the next concert he attends (Tom Petty)
and is greatly surprised by what he finds.

\item A high school principal conducts a survey of how satisfied students are
with his high school by asking students in detention to fill out a
satisfaction survey. Generalizing from that sample, he infers that 79\%
of students are dissatisfied with their high school experience. He is
surprised and saddened by the result.

\item After having attended numerous Pistons home games over 20 years,
Alice cannot remember a time when she didn't see ticket scalpers
selling tickets outside the stadium. She generalizes that there are
always scalpers at every Pistons home game.

\item After having attended numerous Pistons home games over 20 years,
Alice cannot remember a time when she didn't see ticket scalpers
selling tickets outside the stadium. She generalizes that there are
ticket scalpers at every NBA game.

\item After having attended numerous Pistons home games over 20 years,
Alice cannot remember a time when she didn't see ticket scalpers
selling tickets outside the stadium. She generalizes that there are
ticket scalpers at every sporting event.

\item Bob once ordered a hamburger from Burger King and got violently ill
shortly after he ate it. From now on, he never eats at Burger King
because he fears he will get food poisoning.
\end{enumerate}
