%this chapter takes sections from "fundamentals" and also "introduction", Chapter 4, 186-194
\chapter{Fallacies of Illicit Presumption}
This is a family of fallacies whose common characteristic is that they (often tacitly, implicitly)
presume the truth of some claim that they're not entitled to. They are arguments with a premise
(again, often hidden) that is assumed to be true, but is actually a controversial claim, which at best
requires support that's not provided, which at worst is simply false. We will look at six fallacies
under this heading.

\subsubsection{Accident}
This fallacy is the reverse of the hasty generalization. That was a fallacious inference from
insufficient particular premises to a general conclusion; accident is a fallacious inference from a
general premise to a particular conclusion. What makes it fallacious is an illicit presumption: the
general rule in the premise is assumed, incorrectly, not to have any exceptions; the particular
conclusion fallaciously inferred is one of the exceptional cases.

Here's a simple example to help make that clear:

\begin{quote}Cutting people with knives is illegal. \\
\underline{Surgeons cut people with knives.} \\
Surgeons should be arrested. \end{quote}

One of the premises is the general claim that cutting people with knives is illegal. While this is
true in almost all cases, there are exceptions--surgery among them. We pay surgeons lots of
money to cut people with knives! It is therefore fallacious to conclude that surgeons should be
arrested, since they are an exception to the general rule. The inference only goes through if we
presume, incorrectly, that the rule is exceptionless.

Another example. Suppose I volunteer at my first grade daughter's school; I go in to her class one
day to read a book aloud to the children. As I'm sitting down on the floor with the kiddies, 
crisscross applesauce, as they say, I realize that I can't comfortably sit that way because of the .44
Magnum revolver that I have tucked into my waistband.\footnote{That's Dirty Harry's gun, ``the most powerful handgun in the world."} 
So I remove the piece from my pants
and set it down on the floor in front of me, among the circled-up children. The teacher screams
and calls the office, the police are summoned, and I'm arrested. As they're hauling me out of the
room, I protest: ``The Second Amendment to the Constitution guarantees my right to keep and bear
arms! This state has a `concealed carry' law, and I have a license to carry that gun! Let me go!"

I'm committing the fallacy of Accident in this story. True, the Second Amendment guarantees the
right to keep and bear arms; but that rule is not without exceptions. Similarly, concealed carry laws
also have exceptions--among them being a prohibition on carrying weapons into elementary
schools. My insistence on being released only makes sense if we presume, incorrectly, that the
legal rules I'm citing are without exception.

One more example from real life. After the financial crisis in 2008, the Federal Reserve--the
central bank in the United States, whose task it is to create conditions leading to full employment
and moderate inflation--found itself in a bind. The economy was in a free-fall, and unemployment
rates were skyrocketing, but the usual tool it used to mitigate such problems--cutting the shortterm federal funds rate (an interest rate banks charge each other for overnight loans)--was
unavailable, because they had already cut the rate to zero (the lowest it could go). So they had to
resort to unconventional monetary policies, among them something called ``quantitative easing".
This involved the purchase, by the Federal Reserve, of financial assets like mortgage-backed
securities and longer-term government debt 
(Treasury notes).\footnote{The hope was to push down interest rates on mortgages and government debt, encouraging people to buy houses
and spend money instead of saving it--thus stimulating the economy.}

Now, the nice thing about being the Federal Reserve is that when you want to buy something--in
this case a bunch of financial assets--it's really easy to pay for it: you have the power to create
new money out of thin air! That's what the Federal Reserve does; it controls the amount of money
that exists. So if the Fed wants to buy, say, \$10 million worth of securities from Bank of America,
they just press a button and presto--\$10 million dollars that didn't exist a second ago comes into
being as an asset of Bank of 
America.\footnote{It's obviously a bit more complicated than that, but that's the essence of it.}

This quantitative easing policy was controversial. Many people worried that it would lead to
runaway inflation. Generally speaking, the more money there is, the less each bit of it is worth. So
creating more money makes things cost more--inflation. The Fed was creating money on a very
large scale--on the order of a trillion dollars. Shouldn't that lead to a huge amount of inflation?

Economist Art Laffer thought so. In June of 2009, he wrote an op-ed in the Wall Street Journal
warning that ``[t]he unprecedented expansion of the money supply could make the '70s look
benign."\footnote{Art Laffer, ``Get Ready for Inflation and Higher Interest Rates," June 11, 2009, Wall Street Journal}
(There was a lot of inflation in the '70s.)

Another famous economist, Paul Krugman, accused Laffer of committing the fallacy of accident.
While it's generally true that an increase in the supply of money leads to inflation, that rule is not
without exceptions. He had described such exceptional circumstances in 
1998\footnote{``But if current prices are not downwardly flexible, and the public expects price stability in the long run, the economy
cannot get the expected inflation it needs; and in that situation the economy finds itself in a slump against which shortrun monetary expansion, no matter how large, is ineffective." From Paul Krugman, "It's baack: Japan's Slump and the
Return of the Liquidity Trap," 1998, Brookings Papers on Economic Activity, 2}, 
and pointed out
that the economy of 2009 was in that condition (which economists call a ``liquidity trap"): ``Let me
add, for the 1.6 trillionth time, we are in a liquidity trap. And in such circumstances a rise in the
monetary base does not lead to 
inflation."\footnote{Paul Krugman, June 13, 2009, The New York Times}

It turns out Krugman was correct. The expansion of the monetary supply did not lead to runaway
inflation; as a matter of fact, inflation remained below the level that the Federal Reserve wanted,
barely moving at all. Laffer had indeed committed the fallacy of accident.

\subsubsection{Begging the Question (Petitio Principii)}
First things first: `begging the question' is not synonymous with `raising the question'; this is an
extremely common usage, but it is wrong. You might hear a newscaster say, ``Today Donald
Trump's private jet was spotted at the Indianapolis airport, which begs the question: `Will he
choose Indiana Governor Mike Pence as running mate?'" This is a mistaken usage of `begs the
question'; the newscaster should have said `raises the question' instead.

'Begging the question' is a translation of the Latin `petitio principii', which refers to the practice
of asking (begging, petitioning) your audience to grant you the truth of a claim (principle) as a
premise in an argument--but it turns out that the claim you're asking for is either identical to, or
presupposes the truth of, the very conclusion of the argument you're trying to make.

In other words, when you beg the question, you're arguing in a circle: one of the reasons for
believing the conclusion is the conclusion itself! It's a Fallacy of Illicit Presumption where the
proposition being presumed is the very proposition you're trying to demonstrate; that's clearly an
illicit presumption.

Here's a stark example. If I'm trying to convince you that Donald Trump is a dangerous idiot (the
conclusion of my argument is `Donald Trump is a dangerous idiot'), then I can't ask you to grant
me the claim `Donald Trump is a dangerous idiot'. The premise can't be the same as the conclusion.
Imagine a conversation:

\begin{quote}
Me: ``Donald Trump is a dangerous idiot." \\
You: ``Really? Why do you say that?" \\
Me: ``Because Donald Trump is a dangerous idiot." \\
You: ``So you said. But why should I agree with you? Give me some reasons." \\
Me: ``Here's a reason: Donald Trump is a dangerous idiot."
\end{quote}

And round and round we go. Circular reasoning; begging the question.

It's not always so blatant. Sometimes the premise is not identical to the conclusion, but merely
presupposes its truth. Why should we believe that the Bible is true? Because it says so right there
in the Bible that it's the infallible Word of God. This premise is not the same as the conclusion,
but it can only support the conclusion if we take the Bible's word for its own truthfulness, i.e., if
we assume that the Bible is true. But that was the very claim we were trying to prove!

Sometimes the premise is just a re-wording of the conclusion. Consider this argument: ``To allow
every man unbounded freedom of speech must always be, on the whole, advantageous to the state;
for it is highly conducive to the interests of the community that each individual should enjoy a
liberty, perfectly unlimited, of expressing his 
sentiments."\footnote{This is a classic example, from Richard Whately's 1826 Elements of Logic.}
Replacing synonyms with synonyms,
this comes down to ``Free speech is good for society because free speech is good for society." Not
a good argument.\footnote{Though it's valid! P, therefore P is a valid form: if the premise is true, the conclusion must be; they're the same.}

%%%%%%here's the bit from "intro" pg 189
Consider the following argument:

\begin{quote}      Capital punishment is justified for crimes such as rape and murder
      because it is quite legitimate and appropriate for the state to put to
      death someone who has committed such heinous and inhuman acts.\end{quote}

The premise indicator, ``because" denotes the premise and (derivatively) the
conclusion of this argument. In standard form, the argument is this:

\begin{quote}\underline{It is legitimate and appropriate for the state to put to death someone
          who commits rape or murder.}\\
      2. Therefore, capital punishment is justified for crimes such as rape and
          murder.\end{quote}

You should notice something peculiar about this argument: the premise is
essentially the same claim as the conclusion. The only difference is that the
premise spells out what capital punishment means (the state putting criminals to
death) whereas the conclusion just refers to capital punishment by name, and
the premise uses terms like ``legitimate" and ``appropriate" whereas the
conclusion uses the related term, ``justified." But these differences don't add up
to any real differences in meaning. Thus, the premise is essentially saying the
same thing as the conclusion. This is a problem: we want our premise to
provide a reason for accepting the conclusion. But if the premise is the same
claim as the conclusion, then it can't possibly provide a reason for accepting the
conclusion! Begging the question occurs when one (either explicitly or
implicitly) assumes the truth of the conclusion in one or more of the premises.
Begging the question is thus a kind of circular reasoning.

One interesting feature of this fallacy is that formally there is nothing wrong with
arguments of this form. Here is what I mean. Consider an argument that
explicitly commits the fallacy of begging the question. For example,

        \begin{quote}\underline{Capital punishment is morally permissible} \\
       Therefore, capital punishment is morally permissible \end{quote}

Now, apply any method of assessing validity to this argument and you will see
that it is valid by any method. If we use the informal test (by trying to imagine
that the premises are true while the conclusion is false), then the argument
passes the test, since any time the premise is true, the conclusion will have to be
true as well (since it is the exact same statement). Likewise, the argument is
valid by our formal test of validity, truth tables. But while this argument is
technically valid, it is still a really bad argument. Why? Because the point of
giving an argument in the first place is to provide some reason for thinking the
conclusion is true for those who don't already accept the conclusion. But if one
doesn't already accept the conclusion, then simply restating the conclusion in a
different way isn't going to convince them. Rather, a good argument will
provide some reason for accepting the conclusion that is sufficiently
independent of that conclusion itself. Begging the question utterly fails to do
this and this is why it counts as an informal fallacy. What is interesting about
begging the question is that there is absolutely nothing wrong with the
argument formally.

Whether or not an argument begs the question is not always an easy matter to
sort out.      As with all informal fallacies, detecting it requires a careful
understanding of the meaning of the statements involved in the argument. Here
is an example of an argument where it is not as clear whether there is a fallacy of
begging the question:

\begin{quote}        Christian belief is warranted because according to Christianity there exists
        a being called ``the Holy Spirit" which reliably guides Christians towards
        the truth regarding the central claims of Christianity.\footnote{This is a much simplified version of the view defended by Christian philosophers such as Alvin
Plantinga. Plantinga defends (something like) this claim in: Plantinga, A. 2000. Warranted
Christian Belief. Oxford, UK: Oxford University Press.}
\end{quote}

One might think that there is a kind of circularity (or begging the question)
involved in this argument since the argument appears to assume the truth of
Christianity in justifying the claim that Christianity is true. But whether or not this
argument really does beg the question is something on which there is much
debate within the sub-field of philosophy called epistemology (``study of
knowledge"). The philosopher Alvin Plantinga argues persuasively that the
argument does not beg the question, but being able to assess that argument
takes patient years of study in the field of epistemology (not to mention a careful
engagement with Plantinga's work). As this example illustrates, the issue of
whether an argument begs the question requires us to draw on our general
knowledge of the world. This is the mark of an informal, rather than formal,
fallacy.
 %%%%%%%%%%end stuff from "intro" pg 191


\subsubsection{Loaded Questions}
Loaded questions are questions the very asking of which presumes the truth of some claim. Asking
these can be an effective debating technique, a way of sneaking a controversial claim into the
discussion without having outright asserted it.

The classic example of a loaded question is, ``Have you stopped beating your wife?" Notice that
this is a yes-or-no question, and no matter which answer one gives, one admits to beating his wife:
if the answer is `no', then the person continues to beat his wife; if the answer is `yes', then he
admits to beating his wife in the past. Either way, he's a wife-beater. The question itself presumes
the truth of this claim; that's what makes it ``loaded".

Strategic deployment of loaded yes-or-no questions can be an extremely effective debating
technique. If you catch your opponent off-guard, they will struggle to respond to your question,
since a simple `yes' or `no' commits them to the truth of the illicit presumption, which they want
to deny. This makes them look evasive, shifty. And as they struggle to come up with a response,
you can pounce on them: ``It's a simple question. Yes or no? Why won't you answer the question?"
It's a great way to appear to be winning a debate, even if you don't have a good argument. Imagine
the following dialogue:

\begin{quote}
Liberal TV Host: ``Are you or are you not in favor of the president's plan to force wealthy
business owners to pay their fair share in taxes to protect the vulnerable and aid this nation's
underprivileged?" \\
Conservative Guest: ``Well, I don't agree with the way you've laid out the question. As a
matter of fact\dots" \\
Host: ``It's a simple question. Should business owners pay their fair share; yes or no?" \\
Guest: ``You're implying that the president's plan would correct some injustice. But
corporate taxes are already very \dots" \\
Host: ``Stop avoiding the question! It's a simple yes or no!" \\
\end{quote}

Combine this with the sort of subconscious appeal to force discussed above--yelling, fingerpointing, etc.--and 
the host might come off looking like the winner of the debate, with his
opponent appearing evasive, uncooperative, and inarticulate.

Another use for loaded questions is the particularly sneaky political practice of ``push polling". In
a normal opinion poll, you call people up to try to discover what their views are about the issues.
In a push poll, you call people up pretending to be conducting a normal opinion poll, pretending
only to be interested in discovering their views, but with a different intention entirely: you don't
want to know what their views are; you want to shape their views, to convince them of something.
And you use loaded questions to do it.

A famous example of this occurred during the Republican presidential primary in 2000. George
W. Bush was the front-runner, but was facing a surprisingly strong challenge from the upstart John
McCain. After McCain won the New Hampshire primary, he had a lot of momentum. The next
state to vote was South Carolina; it was very important for the Bush campaign to defeat McCain
there and reclaim the momentum. So they conducted a push poll designed to spread negative
feelings about McCain--by implanting false beliefs among the voting public. ``Pollsters" called
voters and asked, ``Would you be more or less likely to vote for John McCain for president if you
knew he had fathered an illegitimate black child?" The aim, of course, is for voters to come to
believe that McCain fathered an illegitimate black child. But he did no such thing. He and his
wife adopted a daughter, Bridget, from Bangladesh.

A final note on loaded questions: there's a minimal sense in which every question is loaded. The
social practice of asking questions is governed by implicit norms. One of these is that it's only
appropriate to ask a question when there's some doubt about the answer. So every question carries
with it the presumption that this norm is being adhered to, that it's a reasonable question to ask,
that the answer is not certain. One can exploit this fact, again to plant beliefs in listeners' minds
that they otherwise wouldn't hold. In a particularly shameful bit of alarmist journalism, the cover
of the July 1, 2016 issue of Newsweek asks the question, ``Can ISIS Take Down Washington?" The
cover is an alarming, eye-catching shade of yellow, and shows four missiles converging on the
Capitol dome. The simple answer to the question, though, is `no, of course not'. There is no
evidence that ISIS has the capacity to destroy the nation's capital. But the very asking of the
question presumes that it's a reasonable thing to wonder about, that there might be a reason to
think that the answer is `yes'. The goal is to scare readers (and sell magazines) by getting them to
believe there might be such a threat.

\subsubsection{False Choice}
This fallacy occurs when someone tries to convince you of something by presenting it as one of
limited number of options and the best choice among those options. The illicit presumption is that
the options are limited in the way presented; in fact, there are additional options that are not
offered. The choice you're asked to make is a false choice, since not all the possibilities have been
presented.

Most frequently, the number of options offered is two. In this case, you're being presented with a
false dilemma. I manipulate my kids with false choices all the time. My younger daughter, for
example, loves cucumbers; they're her favorite vegetable by far. We have a rule at dinner: you've
got to choose a vegetable to eat. Given her 'druthers, she'd choose cucumber every night. Carrots
are pretty good, too; they're the second choice. But I need her to have some more variety, so I'll
sometimes lie and tell her we're out of cucumbers and carrots, and that we only have two options:
broccoli or green beans, for example. That's a false choice; I've deliberately left out other options.
I give her the false choice as a way of manipulating her into choosing green beans, because I know
she dislikes broccoli.

Politicians often treat us like children, presenting their preferred policies as the only acceptable
choice among an artificially restricted set of options. We might be told, for example, that we need
to raise the retirement age or cut Social Security benefits across the board; the budget can't keep
up with the rising number of retirees. Well, nobody wants to cut benefits, so we have to raise the
retirement age. Bummer. But it's a false choice. There are any number of alternative options for
funding an increasing number of retirees: tax increases, re-allocation of other funds, means-testing
for benefits, etc.

Liberals are often ambivalent about free trade agreements. On the one hand, access to American
markets can help raise the living standards of people from poor countries around the world; on the
other hand, such agreements can lead to fewer jobs for American workers in certain sectors of the
economy (e.g., manufacturing). So what to do? Support such agreements or not? Seems like an
impossible choice: harm the global poor or harm American workers. But it may be a false choice,
as this economist argues:

\begin{quote}But trade rules that are more sensitive to social and equity concerns in the advanced
countries are not inherently in conflict with economic growth in poor countries.
Globalization's cheerleaders do considerable damage to their cause by framing the issue as
a stark choice between existing trade arrangements and the persistence of global poverty.
And progressives needlessly force themselves into an undesirable tradeoff.
… Progressives should not buy into a false and counter-productive narrative that sets the
interests of the global poor against the interests of rich countries' lower and middle classes.
With sufficient institutional imagination, the global trade regime can be reformed to the
benefit of both.\footnote{Dani Rodrik, ``A Progressive Logic of Trade," Project Syndicate, 4/13/2016}
\end{quote}

When you think about it, almost every election in America is a False Choice. With the dominance of the two major political parties, we're normally presented with a stark, sometimes 
unpalatable, choice between only two options: the Democrat or the Republican. But of course if enough people decided to vote for a third-party candidate, that person could win. Such 
candidates do exist. But it's perceived as wasting a vote when you choose someone like that. This fact was memorably highlighted on The Simpsons back in the fall of 1996, before the 
presidential election between Bill Clinton and Bob Dole. In the episode, the diabolical, scheming aliens Kang and Kodos (the green guys with the tentacles and giant heads who drool 
constantly) contrive to abduct the two majorparty candidates and perform a ``bio-duplication" procedure that allows Kang and Kodos to appear as Dole and Clinton, respectively. The 
disguised aliens hit the campaign trail and give speeches, making bizarre campaign 
promises.\footnote{Kodos: ``I am Clin-ton. As overlord, all will kneel trembling before me and obey my brutal command. End
communication."} 
When Homer reveals the subterfuge to a horrified crowd, Kodos taunts the voters: ``It's 
true; we are aliens. But what are you going to do about it? It's a twoparty system. You have to vote for one of us." When a guy in the crowd declares his intention to vote for a 
third-party candidate, Kang responds, ``Go ahead, throw your vote away!" Then Kang and Kodos laugh maniacally. Later, as Marge and Homer--chained together and wearing neckcollars--are 
being whipped by an alien slave-driver, Marge complains and Homer quips, ``Don't blame me; I voted for Kodos."

\subsubsection{Composition}
The fallacy of Composition rests on an illicit presumption about the relationship between a whole
thing and the parts that make it up. This is an intuitive distinction, between whole and parts: for
example, a person can be considered as a whole individual thing; it is made up of lots of parts--
hands, feet, brain, lungs, etc., etc. We commit the fallacy of Composition when we mistakenly
assume that any property that all of the parts share is also a property of the whole. Schematically,
it looks like this:

\begin{quote}
All of the parts of X have property P. \\
\underline{Any property shared by all of the parts of a thing is also a property of the whole.} \\
X has the property P.\end{quote}

The second premise is the illicit presumption that makes this argument go through. It is illicit
because it is simply false: sometimes all the parts of something have a property in common, but
the whole does not have that property.

Consider the 1980 U.S. Men's Hockey Team. They won the gold medal at the Olympics that year,
beating the unstoppable-seeming Russian team in the semifinals. (That game is often referred to
as ``The Miracle on Ice" after announcer Al Michaels' memorable call as the seconds ticked off at
the end: ``Do you believe in miracles? Yes!") Famously, the U.S. team that year was a rag-tag
collection of no-name college guys; the average age on the team was 21, making them the youngest
team ever to compete for the U.S. in the Olympics. The Russian team, on the other hand, was
packed with seasoned hockey veterans with world-class talent.

In this example, the team is the whole, and the individual players on the team are the parts. It's
safe to say that one of the properties that all of the parts shared was mediocrity--at least, by the
standards of international competition at the time. They were all good hockey players, of course--
Division I college athletes--but compared to the Hall of Famers the Russians had, they were
mediocre at best. So, all of the parts have the property of being mediocre. But it would be a mistake
to conclude that the whole made up of those parts--the 1980 U.S. Men's Hockey Team--also had
that property. The team was not mediocre; they defeated the Russians and won the gold medal!
They were a classic example of the whole being greater than the sum of its parts.

%%%%%%%%begin "intro" pg 186
Consider the following argument:

\begin{quote}      Each member on the gymnastics team weighs less than 110 lbs.
      Therefore, the whole gymnastics team weighs less than 110 lbs.\end{quote}

This arguments commits the composition fallacy. In the composition fallacy one
argues that since each part of the whole has a certain feature, it follows that the
whole has that same feature. However, you cannot generally identify any
argument that moves from statements about parts to statements about wholes
as committing the composition fallacy because whether or not there is a fallacy
depends on what feature we are attributing to the parts and wholes. Here is an
example of an argument that moves from claims about the parts possessing a
feature to a claim about the whole possessing that same feature, but doesn't
commit the composition fallacy:

\begin{quote}        Every part of the car is made of plastic. Therefore, the whole car is made
        of plastic. \end{quote}

This conclusion does follow from the premises; there is no fallacy here. The
difference between this argument and the preceding argument (about the
gymnastics team) isn't their form. In fact both arguments have the same form:
       
\begin{quote} Every part of X has the feature f. Therefore, the whole X has the feature f.\end{quote}

And yet one of the arguments is clearly fallacious, while the other isn't. The
difference between the two arguments is not their form, but their content. That
is, the difference is what feature is being attributed to the parts and wholes.
Some features (like weighing a certain amount) are such that if they belong to
each part, then it does not follow that they belong to the whole. Other features
(such as being made of plastic) are such that if they belong to each part, it
follows that they belong to the whole.

Here is another example:

\begin{quote}        Every member of the team has been to Paris. Therefore the team has
        been to Paris.\end{quote}

The conclusion of this argument does not follow. Just because each member of
the team has been to Paris, it doesn't follow that the whole team has been to
Paris, since it may not have been the case that each individual was there at the
same time and was there in their capacity as a member of the team. Thus, even
though it is plausible to say that the team is composed of every member of the
team, it doesn't follow that since every member of the team has been to Paris,
the whole team has been to Paris. Contrast that example with this one:

\begin{quote}       Every member of the team was on the plane. Therefore, the whole team
       was on the plane.\end{quote}

This argument, in contrast to the last one, contains no fallacy. It is true that if
every member is on the plane then the whole team is on the plane. And yet
these two arguments have almost exactly the same form. The only difference is
that the first argument is talking about the property, having been to Paris,
whereas the second argument is talking about the property, being on the plane.
The only reason we are able to identify the first argument as committing the
composition fallacy and the second argument as not committing a fallacy is that
we understand the relationship between the concepts involved. In the first case,
we understand that it is possible that every member could have been to Paris
without the team ever having been; in the second case we understand that as
long as every member of the team is on the plane, it has to be true that the
whole team is on the plane. The take home point here is that in order to identify
whether an argument has committed the composition fallacy, one must
understand the concepts involved in the argument. This is the mark of an
informal fallacy: we have to rely on our understanding of the meanings of the
words or concepts involved, rather than simply being able to identify the fallacy
from its form.
%%%%%%%%end "intro" page 188

\subsubsection{Division}
The fallacy of Division is the exact reverse of the fallacy of Composition. It's an inference from
the fact that a whole has some property to a conclusion that a part of that whole has the same
property, based on the illicit presumption that wholes and parts must have the same properties.
Schematically:

\begin{quote}X has the property P. \\
\underline{Any property of a whole thing is shared by all of its parts.} \\
x, which is a part of X, has property P.
\end{quote}

The second premise is the illicit presumption. It is false, because sometimes parts of things don't
have the same properties as the whole. George Clooney is handsome; does it follow that his large
intestine is also handsome? Of course not. Toy Story 3 is a funny movie. Remember when Mr.
Potato Head had to use a tortilla for his body? Or when Buzz gets flipped into Spanish mode and
does the flamenco dance with Jessie? Hilarious. But not all of the parts of the movie are funny.
When it looks like all the toys are about to be incinerated at the dump? When Andy finally drives
off to college? Not funny at all!

%begin bit from "intro" page 188
The division fallacy is like the composition fallacy and they are easy to confuse.
The difference is that the division fallacy argues that since the whole has some
feature, each part must also have that feature. The composition fallacy, as we
have just seen, goes in the opposite direction: since each part has some feature,
the whole must have that same feature. Here is an example of a division fallacy:
        
\begin{quote}
The house costs 1 million dollars. Therefore, each part of the house costs 1 million dollars.
\end{quote}

This is clearly a fallacy. Just because the whole house costs 1 million dollars, it
doesn't follow that each part of the house costs 1 million dollars. However, here
is an argument that has the same form, but that doesn't commit the division
fallacy:

\begin{quote}
        The whole team died in the plane crash. Therefore each individual on the
        team died in the plane crash.
\end{quote}

In this example, since we seem to be referring to one plane crash in which all the
members of the team died (``the" plane crash), it follows that if the whole team
died in the crash, then every individual on the team died in the crash. So this
argument does not commit the division fallacy. In contrast, the following
argument has exactly the same form, but does commit the division fallacy:

\begin{quote}        The team played its worst game ever tonight. Therefore, each individual
        on the team played their worst game ever tonight.\end{quote}

It can be true that the whole team played its worst game ever even if it is true
that no individual on the team played their worst game ever. Thus, this
argument does commit the fallacy of division even though it has the same form
as the previous argument, which doesn't commit the fallacy of division. This
shows (again) that in order to identify informal fallacies (like composition and
division), we must rely on our understanding of the concepts involved in the
argument. Some concepts (like ``team" and ``dying in a plane crash") are such
that if they apply to the whole, they also apply to all the parts. Other concepts
(like ``team" and ``worst game played") are such that they can apply to the
whole even if they do not apply to all the parts.
 %%%%end from "intro page 189

%%%%
\subsubsection{Equivocation}
Typical of natural languages is the phenomenon of homonymy24: when words have the same
spelling and pronunciation, but different meanings--like `bat' (referring to the nocturnal flying
mammal) and `bat' (referring to the thing you hit a baseball with). This kind of natural-language
messiness allows for potential fallacious exploitation: a sneaky debater can manipulate the
subtleties of meaning to convince people of things that aren't true--or at least not justified based
on what they say. We call this kind of maneuver the fallacy of equivocation

Here's an example. Consider a banker; let's call him Fred. Fred is the president of a bank, a real
big-shot. He's married, but he's not faithful: he's carrying on an affair with one of the tellers at his
bank, Linda. Fred and Linda have a favorite activity: they take long lunches away from their
workplace, having romantic picnics at a beautiful spot they found a short walk away. They lay out
their blanket underneath an old, magnificent oak tree, which is situated right next to a river, and
enjoy champagne and strawberries while canoodling and watching the boats float by.
One day--let's say it's the anniversary of when they started their affair--Fred and Linda decide
to celebrate by skipping out of work entirely, spending the whole day at their favorite picnic spot.
(Remember, Fred's the boss, so he can get away with this.) When Fred arrives home that night,
his wife is waiting for him. She suspects that something is up: ``What are you hiding, Fred? Are
you having an affair? I called your office twice, and your secretary said you were `unavailable'
both times. Tell me this: Did you even go to work today?'' Fred replies, ``Scout's honor, dear. I
swear I spent all day at the bank today.''

See what he did there? `Bank' can refer either to a financial institution or the side of a river--a
river bank. Fred and Linda's favorite picnic spot is on a river bank, and Fred did indeed spend the
whole day at that bank. He's trying to convince his wife he hasn't been cheating on her, and he
exploits this little quirk of language to do so. That's equivocation.

Consider the following argument:

 \begin{quote}
       \underline{Children are a headache. Aspirin will make headaches go away}. \\
        Therefore, aspirin will make children go away.
\end{quote}

This is a silly argument, but it illustrates the fallacy of equivocation. The problem
is that the word ``headache'' is used equivocally--that is, in two different senses.
In the first premise, ``headache'' is used figuratively, whereas in the second
premise ``headache'' is used literally. The argument is only successful if the
meaning of ``headache'' is the same in both premises. But it isn't and this is
what makes this argument an instance of the fallacy of equivocation.
Here's another example:

\begin{quote}
        Taking a logic class helps you learn how to argue. But there is already
        too much hostility in the world today, and the fewer arguments the better.
        Therefore, you shouldn't take a logic class.
\end{quote}

In this example, the word ``argue'' and ``argument'' are used equivocally.
Hopefully, at this point in the text, you recognize the difference. (If not, go back
and reread section 1.1.)

The fallacy of equivocation is not always so easy to spot. Here is a trickier
example. A common argument for the existence of God relies on equivocation between these two senses of
`law': \\
\begin{quote}
        There are laws of nature. \\
        By definition, laws are rules imposed by an Authority. \\
        So the laws of nature were imposed by an Authority. \\
        \underline{The only Authority who could impose such laws is an all-powerful Creator--God}. \\
        God exists.
\end{quote}

This argument relies on fallaciously equivocating between the two senses of `law'--human and
natural. It's true that human laws are by definition imposed by an authority; but that is not true of
natural laws. Additional argument is needed to establish that those must be so imposed.

As with every informal fallacy we have examined in this section, equivocation
can only be identified by understanding the meanings of the words involved. In
fact, the definition of the fallacy of equivocation refers to this very fact: the same
word is being used in two different senses (i.e., with two different meanings). So,
unlike formal fallacies, identifying the fallacy of equivocation requires that we
draw on our understanding of the meaning of words and of our understanding
of the world, generally.




%%%%
\subsubsection{Accent}

This is one of the original 13 fallacies that Aristotle recognized in his Sophistical Refutations. Our
usage, however, will depart from Aristotle's. He identifies a potential for ambiguity and
misunderstanding that is peculiar to his language--ancient Greek. That language--in written
form--used diacritical marks along with the alphabet, and transposition of these could lead to
changes in meaning. English is not like this, but we can identify a fallacy that is roughly in line
with the spirit of Aristotle's accent: it is possible, in both written and spoken English (along with
every other language), to convey different meanings by stressing individual words and phrases.
The devious use of stress to emphasize contents that are helpful to one's rhetorical goals, and to
suppress or obscure those that are not--that is the fallacy of accent.

There are a number of techniques one can use with the written word that fall in the category of
accent. Perhaps the simplest way to emphasize favorable contents, and de-emphasize unfavorable
ones, is to vary the size of one's text. We see this in advertising all the time. You drive past a store
that's having a sale, which they advertise with a sign in the window. In the largest, most eye-
catching font, you read, ``70\% 
OFF!'' ``Wow,'' you might think, ``that's a really steep discount. I
should go in to the store and get a great deal.'' At least, that's what the store wants you to think.
They're emphasizing the fact of (at least one) steep discount. If you look more closely at the sign,
however, you'll see the things that they're legally required to say, but that they'd like to de-
emphasize. There's a tiny `Up to' in front of the gigantic `70\% 
OFF!'. For all you know, there's
one crappy item that nobody wants, tucked in the back of the store, that's discounted at 70%;
everything else has much smaller discounts, or none at all. Also, if you squint really hard, you'll
see an asterisk after the `70\% 
OFF!', which leads to some text at the bottom of the poster, in the
tiniest font possible, that reads, ``While supplies last. See store details. Not available in all
locations. Offer not valid weekends or holidays. All sales are final.'' This is the proverbial ``fine
print''. It makes the sale look a lot less exciting. So they hide it.

Footnotes are generally a good place to hide unfavorable content. We all know that CEOs of big
companies--especially banks--get paid ridiculous sums of money. Some of it is just their salary
and stock options; those amounts are huge enough to turn most people off. But there are other
perks that are so over-the-top, companies and executives feel like it's best to hide them from the
public (and their shareholders) in the footnotes of CEO contracts and SEC reports. Michelle Leder
runs a website called footnoted.com, which is dedicated to combing through these documents and
exposing outrageous compensation packages. She's uncovered executives spending over \$700,000
to renovate their offices, demanding helicopters in addition to their corporate jets, receiving
millions of dollars' worth of private security services, etc., etc. These additional, extravagant forms
of compensation seem excessive to most people, so companies do all they can to hide them from
the public.

Another abuse of footnotes can occur in academic or legal writing. Legal briefs and opinions and
academic papers seek to persuade. If you're writing such a document, and you relegate a strong
objection to your conclusion to a brief mention in the footnotes23, you're de-emphasizing that point
of view and making it less likely that the reader will reject your arguments. That's a fallacious
suppression of opposing content, a sneaky trick to try to convince people you're right without
giving them a forthright presentation of the merits (and demerits) of your position.

The fallacy of accent can occur in speech as well as writing. The audible correlate of ``fine print''
is that guy talking really fast at the end of the commercial, rattling off all the unpleasant side effects
and legal disclaimers that, if given a full, deliberate presentation might make you less likely to buy
the product they're selling. The reason, by the way, that we know about such horrors as the
possibility of driving while not awake (a side-effect of some sleep aids) and a four-hour erection
(side-effect of erectile-dysfunction drugs), is that drug companies are required, by federal law, not
to commit the fallacy of accent if they want to market drugs directly to consumers. They have to
read what's called a ``major statement'' that lists all of these side-effects explicitly, and no fair
cramming them in at the end and talking over them really fast.
When we speak, how we stress individual words and phrases can alter the meaning that we convey
with our utterances. Consider the sentence `We should not steal our neighbor's car.' Now consider
various utterances of that sentence, each stressing a different word; different meanings will be
conveyed:

\begin{enumerate}
\item \emph{We} should not steal our neighbor's car.
\item We \emph{shouldn't} steal our neighbor's car.
\item We should not \emph{steal} our neighbor's car.
\item We should not steal \emph{our} neighbor's car.
\item We should not steal our \emph{neighbor's} car.
\item We should not steal our neighbor's \emph{car}.
\end{enumerate}

Try saying each of the above sentences out loud, giving special emphasis on a different word each time, and note the change in meaning when you do.
By stressing a different word, you change the focus, and thus the meaning, of the sentences. To turn an argument on the ambiguity captured in two 
or more sentences such as these risks committing the fallacy of accent.

%%%%
\subsubsection{Amphiboly}


Finally, the fallacy of amphiboly comes about due to an ambiguity that is attributable to the poor grammatical structure of the sentence.
In particular, this will come about when the poor grammatical structure causes the sentences to sound strong and logical, when in fact it is not.

Here's an example: 

\begin{quote}
I'm going to return this car to the
dealer I bought this car from. Their ad said ``Used 1995
Ford Taurus with air conditioning, cruise, leather, new
exhaust and chrome rims.'' But the chrome rims aren't
new at all.
\end{quote}

Here, the argument turns on the grammatical ambiguity of the scope of the term ``new''. Should it be read as including only the exhaust, or also 
the chrome rims? From the grammar alone, it is impossible to tell. However, from the context, it is probably clear that chrome rims on a 1995 Ford 
Taurus will not be new, even if the exhaust system is.

Here's another: 
\begin{quote}
I took some pictures of some kids playing basketball today at the park, but they weren't any good.\end{quote}

From the grammar alone, it's not possible to tell whether the dogs were any good. However, from the context of the speaker, it may be clear. Thus, 
if we were to make some inference, such as for example, ``therefore, those kids should have basketball lessons," we might be making a fallacious 
inference based on the poor grammatical structure of the original sentence.

Let me end with a famous joke from Groucho Marx: ``\emph{One morning I shot an elephant in my pajamas.
How he got into my pajamas I'll never know.}"









