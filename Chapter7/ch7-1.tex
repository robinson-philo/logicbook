
\chapter{Logical Fallacies: Formal and Informal}
Generally and crudely speaking, a logical fallacy is just a bad argument. Bad, that is, in the logical
sense of being incorrect--not bad in sense of being ineffective or unpersuasive. Alas, many
fallacies are quite effective in persuading people; that is why they're so common. Often, they're
not used mistakenly, but intentionally--to fool people, to get them to believe things that maybe
they shouldn't. The goal of this chapter is to develop the ability to recognize these bad arguments
for what they are so as not to be persuaded by them.

There are formal and informal logical fallacies. The formal fallacies are simple: they're just invalid
deductive arguments. Consider the following:

\begin{quote}
If the Democrats retake Congress, then taxes will go up. \\
\underline{But the Democrats won't retake Congress.}
Taxes won't go up.
\end{quote}

This argument is invalid. It's got an invalid form: If A then B; not A; therefore, not B. Any
argument of this form is fallacious, an instance of ``Denying the 
Antecedent."\footnote{If/then propositions like the first premise are called ``conditional" 
propositions. The A part is the so-called
``antecedent" of the conditional. The second premise denies it.} 
We can leave it as
an exercise for the reader to fill in propositions for A and B to get true premises and a false
conclusion. Intuitively, it's possible for that to happen: maybe a Republican Congress raises taxes
for some reason (unlikely, but not unprecedented).

Our concern in this chapter is not with formal fallacies--arguments that are bad because they have
a bad form--but with informal fallacies. These arguments are bad, roughly, because of their
content. More than that: their content, context, and/or mode of delivery.

Consider Hitler. Here's a guy who convinced a lot of people to believe things they had no business
believing (because they were false). How did he do it? With lots of fallacious arguments. But it
wasn't just the contents of the arguments (appeals to fear and patriotism, personal attacks on
opponents, etc.) that made them fallacious; it was also the context in which he made them, and the
(extremely effective) way he delivered them. Leni Riefenstahl's famous 1935
documentary/propaganda film Triumph of the Will, which follows Hitler during a Nazi party rally
in Nuremberg, illustrates this. It has lots of footage of Hitler giving speeches. We hear the jingoistic
slogans and vitriolic attacks--but we also see important elements of his persuasive technique.
First, the setting. We see Hitler marching through row upon row of neatly formed and impeccably
outfitted German troops--thousands of them--approaching a massive raised dais, behind which
are stories-high banners with the swastika on a red field. The setting, the context for Hitler's
speeches, was literally awesome--designed to inspire awe. It makes his audience all the more
receptive to his message, all the more persuadable. Moreover, Hitler's speechifying technique was
masterful. He is said to have practiced assiduously in front of a mirror, and it shows. His array of
hand gestures, facial contortions, and vocal modulations were all expertly designed to have
maximum impact on the audience.

This consideration of Hitler highlights a couple of important things about the informal fallacies.
First, they're more than just bad arguments--they're rhetorical tricks, extra-logical techniques
used intentionally to try to convince people of things they maybe ought not to believe. Second,
they work! Hitler convinced an entire nation to believe all sorts of crazy things. And advertisers
and politicians continue to use these same techniques all the time. It's incumbent upon a
responsible citizen and consumer to be aware of this, and to do everything possible to avoid being
bamboozled. That means learning about the fallacies. Hence, this chapter.

There are lots of different informal logical fallacies, lots of different ways of defining and
characterizing them, lots of different ways of organizing them into groups. Since Aristotle first did
it in his Sophistical Refutations, authors of logic books have been defining and classifying the
informal fallacies in various ways. These remarks are offered as a kind of disclaimer: the reader is
warned that the particular presentation of the fallacies in this chapter will be unique and will
disagree in various ways with other presentations, reflecting as it must the author's own
idiosyncratic interests, understanding, and estimation of what is important. This is as it should be
and always is. The interested reader is encouraged to consult alternative sources for further
edification.

We will discuss many different informal fallacies, and we will group them into four families: (1)
Fallacies of Distraction, (2) Fallacies of Weak Induction, (3) Fallacies of Illicit Presumption, and
(4) Fallacies of Linguistic Emphasis. We take these up in turn.

\subsection{Fallacies of Distraction}
We will discuss five informal fallacies under this heading. What they all have in common is that
they involve arguing in such a way that issue that's supposed to be under discussion is somehow
sidestepped, avoided, or ignored. These fallacies are often called ``Fallacies of Relevance" because
they involve arguments that are bad insofar as the reasons given are irrelevant to the issue at hand.
People who use these techniques with malicious intent are attempting to distract their audience
from the central questions they're supposed to be addressing, allowing them to appear to win an
argument that they haven't really engaged in.

\paragraph{Appeal to Emotion (Argumentum ad Populum)}


The Latin name\footnote{Many of the fallacies have 
Latin names, because, as we noted, identifying the fallacies has been an occupation of
logicians since ancient times, and because ancient and medieval work comes down to us in Latin, which was the
language of scholarship in the West for centuries.}
of this fallacy literally means ``argument to the people," where `the people' is used
in the pejorative sense of ``the unwashed masses," or ``the fickle mob"--the hoi polloi. It's
notoriously effective to play on people's emotions to get them to go along with you, and that's the
technique identified here. But, the thought is, we shouldn't decide whether or not to believe things
based on an emotional response; emotions are a distraction, blocking hard-headed, rational
analysis.

Go back to Hitler for a minute. He was an expert at the appeal to emotion. He played on Germans'
fears and prejudices, their economic anxieties, their sense of patriotism and nationalistic pride. He
stoked these emotions with explicit denunciations of Jews and non-Germans, promises of the
return of glory for the Fatherland--but also using the sorts of techniques we canvassed above, with
awesome settings and hyper-sensational speechifying.

There are as many different versions of the appeal to emotion as there are human emotions. Fear
is perhaps the most commonly exploited emotion for politicians. Political ads inevitably try to
suggest to voters that one's opponent will take away medical care or leave us vulnerable to
terrorists, or some other scary outcome--usually without a whole lot in the way of substantive
proof that these fears are at all reasonable. This is a fallacious appeal to emotion.

Advertisers do it, too. Think of all the ads with sexy models schilling for cars or beers or whatever.
What does sexiness have to do with how good a beer tastes? Nothing. The ads are trying to engage
your emotions to get you thinking positively about their product.

An extremely common technique, especially for advertisers, is to appeal to people's underlying
desire to fit in, to be hip to what everybody else is doing, not to miss out. This is the bandwagon
appeal. The advertisement assures us that a certain television show is number 1 in the ratings--with the
tacit conclusion being that we should be watching, too. But this is a fallacy. We've all known it's
a fallacy since we were little kids, the first time we did something wrong because all of our friends
were doing it, too, and our moms asked us, ``If all of your friends jumped off a bridge, would you
do that too?"

One more example: suppose you're one of those sleazy personal injury lawyers--an ``ambulance
chaser". You've got a client who was grocery shopping at Wal-Mart, and in the produce aisle she
slipped on a grape that had fallen on the floor and injured herself. Your eyes turn into dollar signs
and a cha-ching noise goes off in your brain: Wal-Mart has deep pockets. So on the day of the
trial, what do you do? How do you coach your client? Tell her to wear her nicest outfit, to look her
best? Of course not! You wheel her into the courtroom in a wheelchair (whether she needs it or
not); you put one of those foam neck braces on her, maybe give her an eye patch for good measure.
You tell her to periodically emit moans of pain. When you're summing up your case before the
jury, you spend most of your time talking about the horrible suffering your client has undergone
since the incident in the produce aisle: the hospital stays, the grueling physical therapy, the
addiction to pain medications, etc., etc.

All of this is a classic fallacious appeal to emotion--specifically, in this case, pity. The people
you're trying to convince are the jurors. The conclusion you have to convince them of, presumably,
is that Wal-Mart was negligent and hence legally liable in the matter of the grape on the floor. The
details don't matter, but there are specific conditions that have to be met--proved beyond a
reasonable doubt--in order for the jury to find Wal-Mart guilty. But you're not addressing those
(probably because you can't). Instead, you're trying to distract the jury from the real issue by
playing to their emotions. You're trying to get them feeling sorry for your client, in the hopes that
those emotions will cause them to bring in the verdict you want. That's why the appeal to emotion
is a Fallacy of Distraction: the goal is to divert your attention from the dispassionate evaluation of
premises and the degree to which they support their conclusion, to get you thinking with your heart
instead of your brain.

Appeal to Force (Argumentum ad Baculum\footnote{In Latin, `baculus' refers to a stick or a club, 
which you could clobber someone with, presumably.})

Perhaps the least subtle of the fallacies is the appeal to force, in which you attempt to convince
your interlocutor to believe something by threatening him. Threats pretty clearly distract one from
the business of dispassionately appraising premises' support for conclusions, so it's natural to
classify this technique as a Fallacy of Distraction.

There are many examples of this technique throughout history. In totalitarian regimes, there are
often severe consequences for those who don't toe the party line (see George Orwell's 1984 for a
vivid, though fictional, depiction of the phenomenon). The Catholic Church used this technique
during the infamous Spanish Inquisition: the goal was to get non-believers to accept Christianity;
the method was to torture them until they did.

An example from much more recent history: when it became clear in 2016 that Donald Trump
would be the Republican nominee for president, despite the fact that many rank-and-file
Republicans thought he would be a disaster, the Chairman of the Republican National Committee
(allegedly) sent a message to staffers informing them that they could either support Trump or leave
their jobs. Not a threat of physical force, but a threat of being fired; same technique.

Again, the appeal to force is not usually subtle. But there is a very common, very effective debating
technique that belongs under this heading, one that is a bit less overt than explicitly threatening
someone who fails to share your opinions. It involves the sub-conscious, rather than conscious,
perception of a threat.

Here's what you do: during the course of a debate, make yourself physically imposing; sit up in
your chair, move closer to your opponent, use hand gestures, like pointing right in their face; cut
them off in the middle of a sentence, shout them down, be angry and combative. If you do these
things, you're likely to make your opponent very uncomfortable--physically and emotionally.
They might start sweating a bit; their heart may beat a little faster. They'll get flustered and maybe
trip over their words. They may lose their train of thought; winning points they may have made in
the debate will come out wrong or not at all. You'll look like the more effective debater, and the
audience's perception will be that you made the better argument.

But you didn't. You came off better because your opponent was uncomfortable. The discomfort
was not caused by an actual threat of violence; on a conscious level, they never believed you were
going to attack them physically. But you behaved in a way that triggered, at the sub-conscious
level, the types of physical/emotional reactions that occur in the presence of an actual physical
threat. This is the more subtle version of the appeal to force. It's very effective and quite common
(watch cable news talk shows and you'll see it; Bill O'Reilly is the master).

\paragraph{Straw Man}

This fallacy involves the misrepresentation of an opponent's viewpoint--an exaggeration or
distortion of it that renders it indefensible, something nobody in their right mind would agree with.
You make your opponent out to be a complete wacko (even though he isn't), then declare that you
don't agree with his (made-up) position. Thus, you merely appear to defeat your opponent: your
real opponent doesn't hold the crazy view you imputed to him; instead, you've defeated a distorted
version of him, one of your own making, one that is easily dispatched. Instead of taking on the real
man, you construct one out of straw, thrash it, and pretend to have achieved victory. It works if
your audience doesn't realize what you've done, if they believe that your opponent really holds
the crazy view.

Politicians are most frequently victims (and practitioners) of this tactic. After his 2005 State of the
Union Address, President George W. Bush's proposals were characterized thus:

\begin{quote}George W. Bush's State of the Union Address, masked in talk of "freedom" and
"democracy," was an outline of a brutal agenda of endless war, global empire, and the
destruction of what remains of basic social 
services.\footnote{International Action Center, Feb. 4 2005, http://iacenter.org/folder06/stateoftheunion.htm}
\end{quote}

Well who's not against ``endless war" and ``destruction of basic social services"? That Bush guy
must be a complete nut! But of course this characterization is a gross exaggeration of what was
actually said in the speech, in which Bush declared that we must "confront regimes that continue
to harbor terrorists and pursue weapons of mass murder" and rolled out his proposal for
privatization of Social Security accounts. Whatever you think of those actual policies, you need to
do more to undermine them than to mic-characterize them as ``endless war" and ``destruction of
social services." That's distracting your audience from the real substance of the issues.

In 2009, during the (interminable) debate over President Obama's healthcare reform bill--the
Patient Protection and Affordable Care Act--former vice presidential candidate Sarah Palin took
to Facebook to denounce the bill thus:

\begin{quote}The America I know and love is not one in which my parents or my baby with Down
Syndrome will have to stand in front of Obama's "death panel" so his bureaucrats can
decide, based on a subjective judgment of their "level of productivity in society," whether
they are worthy of health care. Such a system is downright evil.
\end{quote}

Yikes! That sounds like the evilest bill in the history of evil! Bureaucrats euthanizing Down
Syndrome babies and their grandparents? Holy Cow. `Death panel' and `level of productivity in
society' are even in quotes. Did she pull those phrases from the text of the bill?
Of course she didn't. This is a completely insane distortion of what's actually in the bill (the kernel
of truth behind the ``death panels" thing seems to be a provision in the Act calling for Medicare to
fund doctor-patient conversations about end-of-life care); the non-partisan fact-checking outfit

Politifact named it their ``Lie of the Year" in 2009. Palin is not taking on the bill or the president
themselves; she's confronting a made-up version, defeating it (which is easy, because the madeup bill is 
evil as heck; I can't get the disturbing idea of a Kafkaesque Death Panel out of my head),
and pretending to have won the debate. But this distraction only works if her audience believes her
straw man is the real thing. Alas, many did. But of course this is why these techniques are used so
frequently: they work.

\paragraph{Red Herring}
This fallacy gets its name from the actual fish. When herring are smoked, they turn red and are
quite pungent. Stinky things can be used to distract hunting dogs, who of course follow the trail of
their quarry by scent; if you pass over that trail with a stinky fish and run off in a different direction,
the hound may be distracted and follow the wrong trail. Whether or not this practice was ever used
to train hunting dogs, as some suppose, the connection to logic and argumentation is clear. One
commits the red herring fallacy when one attempts to distract one's audience from the main thread
of an argument, taking things off in a different direction. The diversion is often subtle, with the
detour starting on a topic closely related to the original--but gradually wandering off into
unrelated territory. The tactic is often (but not always) intentional: one commits the red herring
fallacy because one is not comfortable arguing about a particular topic on the merits, often because
one's case is weak; so instead, the arguer changes the subject to an issue about which he feels more
confident, makes strong points on the new topic, and pretends to have won the original 
argument.\footnote{People often offer red herring arguments unintentionally, without the subtle deceptive motivation to change the
subject--usually because they're just parroting a red herring argument they heard from someone else. Sometimes a
person's response will be off-topic, apparently because they weren't listening to their interlocutor or they're confused
for some reason. I prefer to label such responses as instances of Missing the Point (Ignoratio Elenchi), a fallacy that
some books discuss at length, but which I've just relegated to a footnote.}

A fictional example can illustrate the technique. Consider Frank, who, after a hard day at work,
heads to the tavern to unwind. He has far too much to drink, and, unwisely, decides to drive home.
Well, he's swerving all over the road, and he gets pulled over by the police. Let's suppose that
Frank has been pulled over in a posh suburb where there's not a lot of crime. When the police
officer tells him he's going to be arrested for drunk driving, Frank becomes belligerent:

\begin{quote}``Where do you get off? You're barely even real cops out here in the 'burbs. All you do is
sit around all day and pull people over for speeding and stuff. Why don't you go investigate
some real crimes? There's probably some unsolved murders in the inner city they could
use some help with. Why do you have to bother a hard-working citizen like me who just
wants to go home and go to bed?"
\end{quote}

Frank is committing the red herring fallacy (and not very subtly). The issue at hand is whether or
not he deserves to be arrested for driving drunk. He clearly does. Frank is not comfortable arguing
against that position on the merits. So he changes the subject--to one about which he feels like he
can score some debating points. He talks about the police out here in the suburbs, who, not having
much serious crime to deal with, spend most of their time issuing traffic violations. Yes, maybe
that's not as taxing a job as policing in the city. Sure, there are lots of serious crimes in other
jurisdictions that go unsolved. But that's beside the point! It's a distraction from the real issue of
whether Frank should get a DUI.

Politicians use the red herring fallacy all the time. Consider a debate about Social Security--a
retirement stipend paid to all workers by the federal government. Suppose a politician makes the
following argument:

\begin{quote}
We need to cut Social Security benefits, raise the retirement age, or both. As the baby boom
generation reaches retirement age, the amount of money set aside for their benefits will not
be enough cover them while ensuring the same standard of living for future generations
when they retire. The status quo will put enormous strains on the federal budget going
forward, and we are already dealing with large, economically dangerous budget deficits
now. We must reform Social Security.
\end{quote}

Now imagine an opponent of the proposed reforms offering the following reply:

\begin{quote}
Social Security is a sacred trust, instituted during the Great Depression by FDR to insure
that no hard-working American would have to spend their retirement years in poverty. I
stand by that principle. Every citizen deserves a dignified retirement. Social Security is a
more important part of that than ever these days, since the downturn in the stock market
has left many retirees with very little investment income to supplement government
support.\end{quote}


The second speaker makes some good points, but notice that they do not speak to the assertion
made by the first: Social Security is economically unsustainable in its current form. It's possible
to address that point head on, either by making the case that in fact the economic problems are
exaggerated or non-existent, or by making the case that a tax increase could fix the problems. The
respondent does neither of those things, though; he changes the subject, and talks about the
importance of dignity in retirement. I'm sure he's more comfortable talking about that subject than
the economic questions raised by the first speaker, but it's a distraction from that issue--a red
herring.

Perhaps the most blatant kind of red herring is evasive: used especially by politicians, this is the
refusal to answer a direct question by changing the subject. Examples are almost too numerous to
cite; to some degree, no politician ever answers difficult questions straightforwardly (there's an
old axiom in politics, put nicely by Robert McNamara: ``Never answer the question that is asked
of you. Answer the question that you wish had been asked of you.").

A particularly egregious example of this occurred in 2009 on CNN's Larry King Live. Michele
Bachmann, Republican Congresswoman from Minnesota, was the guest. The topic was
``birtherism," the (false) belief among some that Barack Obama was not in fact born in America
and was therefore not constitutionally eligible for the presidency. After playing a clip of Senator
Lindsey Graham (R, South Carolina) denouncing the myth and those who spread it, King asked
Bachmann whether she agreed with Senator Graham. She responded thus:

\begin{quote}``You know, it's so interesting, this whole birther issue hasn't even been one that's ever been
brought up to me by my constituents. They continually ask me, where's the jobs? That's
what they want to know, where are the jobs?"\end{quote}

Bachmann doesn't want to respond directly to the question. If she outright declares that the
``birthers" are right, she looks crazy for endorsing a clearly false belief. But if she denounces them,
she alienates a lot of her potential voters who believe the falsehood. Tough bind. So she blatantly,
and rather desperately, tries to change the subject. Jobs! Let's talk about those instead. Please?

\paragraph{Argumentum ad Hominem}
Everybody always used the Latin for this one--usually shortened to just `ad hominem', which
means `at the person'. You commit this fallacy when, instead of attacking your opponent's views,
you attack your opponent himself.

This fallacy comes in a lot of different forms; there are a lot of different ways to attack a person
while ignoring (or downplaying) their actual arguments. To organize things a bit, we'll divide the
various ad hominem attacks into two groups: Abusive and Circumstantial.

Abusive ad hominem is the more straightforward of the two. The simplest version is simply calling
your opponent names instead of debating him. Donald Trump has mastered this technique. During
the 2016 Republican presidential primary, he came up with catchy little nicknames for his
opponents, which he used just about every time he referred to them: ``Lyin' Ted" Cruz, ``Little
Marco" Rubio, ``Low-Energy Jeb" Bush. If you pepper your descriptions of your opponent with
tendentious, unflattering, politically charged language, you can get a rhetorical leg-up. Here's
another example, from Wisconsin Supreme Court Justice Rebecca Bradley reacting to the election
of Bill Clinton in her college newspaper:

\begin{quote}
Congratulations everyone. We have now elected a tree-hugging, baby-killing, potsmoking, flag-burning, queer-loving, draft-dodging, bull-spouting '60s radical socialist
adulterer to the highest office in our nation. Doesn't it make you proud to be an American?
We've just had an election which proves that the majority of voters are either totally stupid
or entirely evil.\footnote{Marquette Tribune, 11/11/92}
\end{quote}

Whoa. I guess that one speaks for itself.

Another abusive ad hominem attack is guilt by association. Here, you tarnish your opponent by
associating him or his views with someone or something that your audience despises. Consider the
following:

\begin{quote}
Former Vice President Dick Cheney was an advocate of a strong version of the so-called
Unitary Executive interpretation of the Constitution, according to which the president's
control over the executive branch of government is quite firm and far-reaching. The effect
of this is to concentrate a tremendous amount of power in the Chief Executive, such that
those powers arguably eclipse those of the supposedly co-equal Legislative and Judicial
branches of government. You know who else was in favor of a very strong, powerful Chief
Executive? That's right, Hitler.
\end{quote}


We just compared Dick Cheney to Hitler. Ouch. Nobody likes Hitler, so…. Not every comparison
like this is fallacious, of course. But in this case, where the connection is particularly flimsy, we're
clearly pulling a fast one.\footnote{Comparing your opponent to Hitler--or the Nazis--is quite common. Some clever folks came up with a fake-Latin
term for the tactic: Argumentum ad Nazium (cf. the real Latin phrase, ad nauseum--to the point of nausea). Such
comparisons are so common that author Mike Godwin formulated ``Godwin's Law of Nazi Analogies: As an online
discussion grows longer, the probability of a comparison involving Nazis or Hitler approaches one." (``Meme,
Counter-meme", Wired, 10/1/94)}

The circumstantial ad hominem fallacy is not as blunt an instrument as its abusive counterpart. It
also involves attacking one's opponent, focusing on some aspect of his person--his
circumstances--as the core of the criticism. This version of the fallacy comes in many different
forms, and some of the circumstantial criticisms involved raise legitimate concerns about the
relationship between the arguer and his argument. They only rise (sink?) to the level of fallacy
when these criticisms are taken to be definitive refutations, which, on their own, they cannot be.

To see what we're talking about, consider the circumstantial ad hominem attack that points out
one's opponent's self-interest in making the argument he does. Consider:

\begin{quote}
A recent study from scientists at the University of Minnesota claims to show that
glyphosate--the main active ingredient in the widely used herbicide Roundup--is safe for
humans to use. But guess whose business school just got a huge donation from Monsanto,
the company that produces Roundup? That's right, the University of Minnesota. Ever hear
of conflict of interest? This study is junk, just like the product it's defending.
\end{quote}

This is a fallacy. It doesn't follow from the fact that the University received a grant from Monsanto
that scientists working at that school faked the results of a study. But the fact of the grant does
raise a red flag. There may be some conflict of interest at play. Such things have happened in the
past (e.g., studies funded by Big Tobacco showing that smoking is harmless). But raising the
possibility of a conflict is not enough, on its own, to show that the study in question can be
dismissed out of hand. It may be appropriate to subject it to heightened scrutiny, but we cannot
shirk our duty to assess its arguments on their merits.

A similar thing happens when we point to the hypocrisy of someone making a certain argument--
when their actions are inconsistent with the conclusion they're trying to convince us of. Consider
the following:

\begin{quote}The head of the local branch of the American Federation of Teachers union wrote an op-ed yesterday in which she defended public school teachers from criticism and made the
case that public schools' quality has never been higher. But guess what? She sends her own
kids to private schools out in the suburbs! What a hypocrite. The public school system is a
wreck and we need more accountability for teachers.
\end{quote}

This passage makes a strong point, but then commits a fallacy. It would appear that, indeed, the
AFT leader is hypocritical; her choice to send her kids to private schools suggests (but doesn't
necessarily prove) that she doesn't believe her own assertions about the quality of public schools.
Again, this raises a red flag about her arguments; it's a reason to subject them to heightened
scrutiny. But it is not a sufficient reason to reject them out of hand, and to accept the opposite of
her conclusions. That's committing a fallacy. She may have perfectly good reasons, having nothing
to do with the allegedly low quality of public schools, for sending her kids to the private school in
the suburbs. Or she may not. She may secretly think, deep down, that her kids would be better off
not going to public schools. But none of this means her arguments in the op-ed should be
dismissed; it's beside the point. Do her premises back up her conclusion? Are her premises true?
That's how we evaluate an argument; hypocrisy on the part of the arguer doesn't relieve us of the
responsibility to conduct thorough, dispassionate logical analysis.

A very specific version of the circumstantial ad hominem, one that involves pointing out one's
opponent's hypocrisy, is worth highlighting, since it happens so frequently. It has its own Latin
name: tu quoque, which translates roughly as ``you, too." This is the ``I know you are but what am
I?" fallacy; the ``pot calling the kettle black"; ``look who's talking". It's a technique used in very
specific circumstances: your opponent accuses you of doing or advocating something that's wrong,
and, instead of making an argument to defend the rightness of your actions, you simply throw the
accusation back in your opponent's face--they did it too. But that doesn't make it right!

An example. In February 2016, Supreme Court Justice Antonin Scalia died unexpectedly.
President Obama, as is his constitutional duty, nominated a successor. The Senate is supposed to
`advise and consent' (or not consent) to such nominations, but instead of holding hearings on the
nominee (Merrick Garland), the Republican leaders of the Senate declared that they wouldn't even
consider the nomination. Since the presidential primary season had already begun, they reasoned,
they should wait until the voters has spoken and allow the new president to make a nomination.
Democrats objected strenuously, arguing that the Republicans were shirking their constitutional
duty. The response was classic tu quoque. A conservative writer asked, ``Does any sentient human
being believe that if the Democrats had the Senate majority in the final year of a conservative
president's second term--and Justice [Ruth Bader] Ginsburg's seat came open--they would
approve any nominee from that 
president?"\footnote{David French, National Review, 2/14/16}
Senate Majority Leader Mitch McConnell said that
he was merely following the ``Biden Rule," a principle advocated by Vice President Joe Biden
when he was a Senator, back in the election year of 1992, that then-President Bush should wait
until after the election season was over before appointing a new Justice (the rule was hypothetical;
there was no Supreme Court vacancy at the time).

This is a fallacious argument. Whether or not Democrats would do the same thing if the
circumstances were reversed is irrelevant to determining whether that's the right, constitutional
thing to do.

The final variant of the circumstantial ad hominem fallacy is perhaps the most egregious. It's
certainly the most ambitious: it's a preemptive attack on one's opponent to the effect that, because
of the type of person he is, nothing he says on a particular topic can be taken seriously; he is
excluded entirely from debate. It's called poisoning the well. This phrase was coined by the famous
19th century Catholic intellectual John Henry Cardinal Newman, who was a victim of the tactic. In
the course of a dispute he was having with the famous Protestant intellectual Charles Kingsley,
Kingsley is said to have remarked that anything Newman said was suspect, since, as a Catholic
priest, his first allegiance was not to the truth (but rather to the Pope). As Newman rightly pointed
out, this remark, if taken seriously, has the effect of rendering it impossible for him or any other
Catholic to participate in any debate whatsoever. He accused Kingsley of ``poisoning the wells."

We poison the well when we exclude someone from a debate because of who they are. Imagine an
Englishman saying something like, ``It seems to me that you Americans should reform your
healthcare system. Costs over here are much higher than they are in England. And you have
millions of people who don't even have access to healthcare. In the UK, we have the NHS
(National Health Service); medical care is a basic right of every citizen." Suppose an American
responded by saying, ``What you know about it, Limey? Go back to England." That would be
poisoning the well (with a little name-calling thrown in). The Englishman is excluded from
debating American healthcare just because of who he is--an Englishman, not an American.

