%from ch7-2 pages

\paragraph{Tu quoque}
``Tu quoque" is a Latin phrase that can be translated into English as ``you too"
or ``you, also." The tu quoque fallacy is a way of avoiding answering a criticism
by bringing up a criticism of your opponent rather than answer the criticism. For
example, suppose that two political candidates, A and B, are discussing their
policies and A brings up a criticism of B's policy. In response, B brings up her
own criticism of A's policy rather than respond to A's criticism of her policy. B
has here committed the tu quoque fallacy. The fallacy is best understood as a
way of avoiding having to answer a tough criticism that one may not have a
good answer to. This kind of thing happens all the time in political discourse.

Tu quoque, as I have presented it, is fallacious when the criticism one raises is
simply in order to avoid having to answer a difficult objection to one's argument
or view. However, there are circumstances in which a tu quoque kind of
response is not fallacious. If the criticism that A brings toward B is a criticism
that equally applies not only to A's position but to any position, then B is right to
point this fact out. For example, suppose that A criticizes B for taking money
from special interest groups. In this case, B would be totally right (and there
would be no tu quoque fallacy committed) to respond that not only does A take
money from special interest groups, but every political candidate running for
office does. That is just a fact of life in American politics today. So A really has
no criticism at all to B since everyone does what B is doing and it is in many
ways unavoidable. Thus, B could (and should) respond with a ``you too" rebuttal
and in this case that rebuttal is not a tu quoque fallacy.

\paragraph{Genetic fallacy}
The genetic fallacy occurs when one argues (or, more commonly, implies) that
the origin of something (e.g., a theory, idea, policy, etc.) is a reason for rejecting
(or accepting) it. For example, suppose that Jack is arguing that we should
allow physician assisted suicide and Jill responds that that idea first was used in
Nazi Germany. Jill has just committed a genetic fallacy because she is implying
that because the idea is associated with Nazi Germany, there must be
something wrong with the idea itself. What she should have done instead is
explain what, exactly, is wrong with the idea rather than simply assuming that
there must be something wrong with it since it has a negative origin. The origin
of an idea has nothing inherently to do with its truth or plausibility. Suppose
that Hitler constructed a mathematical proof in his early adulthood (he didn't,
but just suppose). The validity of that mathematical proof stands on its own; the
fact that Hitler was a horrible person has nothing to do with whether the proof is
good. Likewise with any other idea: ideas must be assessed on their own merits
and the origin of an idea is neither a merit nor demerit of the idea.

Although genetic fallacies are most often committed when one associates an
idea with a negative origin, it can also go the other way: one can imply that
because the idea has a positive origin, the idea must be true or more plausible.
For example, suppose that Jill argues that the Golden Rule is a good way to live
one's life because the Golden Rule originated with Jesus in the Sermon on the
Mount (it didn't, actually, even though Jesus does state a version of the Golden
Rule). Jill has committed the genetic fallacy in assuming that the (presumed)
fact that Jesus is the origin of the Golden Rule has anything to do with whether
the Golden Rule is a good idea.

I'll end with an example from William James's seminal work, The Varieties of
Religious Experience. In that book (originally a set of lectures), James considers
the idea that if religious experiences could be explained in terms of neurological
causes, then the legitimacy of the religious experience is undermined. James,
being a materialist who thinks that all mental states are physical states--
ultimately a matter of complex brain chemistry, says that the fact that any
religious experience has a physical cause does not undermine that veracity of
that experience. Although he doesn't use the term explicitly, James claims that
the claim that the physical origin of some experience undermines the veracity of
that experience is a genetic fallacy. Origin is irrelevant for assessing the veracity
of an experience, James thinks. In fact, he thinks that religious dogmatists who
take the origin of the Bible to be the word of God are making exactly the same
mistake as those who think that a physical explanation of a religious experience
would undermine its veracity. We must assess ideas for their merits, James
thinks, not their origins.

\paragraph{Appeal to consequences}
The appeal to consequences fallacy is like the reverse of the genetic fallacy:
whereas the genetic fallacy consists in the mistake of trying to assess the truth or
reasonableness of an idea based on the origin of the idea, the appeal to
consequences fallacy consists in the mistake of trying to assess the truth or
reasonableness of an idea based on the (typically negative) consequences of
accepting that idea. For example, suppose that the results of a study revealed
that there are IQ differences between different races (this is a fictitious example,
there is no such study that I know of). In debating the results of this study, one
researcher claims that if we were to accept these results, it would lead to
increased racism in our society, which is not tolerable. Therefore, these results
must not be right since if they were accepted, it would lead to increased racism.
The researcher who responded in this way has committed the appeal to
consequences fallacy. Again, we must assess the study on its own merits. If
there is something wrong with the study, some flaw in its design, for example,
then that would be a relevant criticism of the study. However, the fact that the
results of the study, if widely circulated, would have a negative effect on society
is not a reason for rejecting these results as false. The consequences of some
idea (good or bad) are irrelevant to the truth or reasonableness of that idea.

Notice that the researchers, being convinced of the negative consequences of
the study on society, might rationally choose not to publish the study (for fear of
the negative consequences). This is totally fine and is not a fallacy. The fallacy
consists not in choosing not to publish something that could have adverse
consequences, but in claiming that the results themselves are undermined by
the negative consequences they could have. The fact is, sometimes truth can
have negative consequences and falsehoods can have positive consequences.
This just goes to show that the consequences of an idea are irrelevant to the
truth or reasonableness of an idea.
